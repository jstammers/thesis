\chapter{Outlook}
\section{Conclusions}
This project set out to explore the use of atom interferometry for
inertial navigation. The specific aim was to demonstrate sensitivity
to horizontal accelerations where gravitational acceleration acts
perpendicularly to the sensitive axis. Requirements for a high
measurement bandwidth and low dead time influenced the development of
the experiment. \par\noindent
The dead time between measurements is reduced by
loading the atoms from a 2D \ac{mot} and through an improved design of
the control software. Using a 2D \ac{mot} reduces the time required to
load a sufficient number of atoms when compared to loading from a
background vapour. The hardware that controls the experiment
regenerates the necessary voltage patterns, removing the need for
additional time to
re-calculate the experimental sequence. This enables the loading of
$10^8$ atoms into the 3D \ac{mot} after \sivalue{100}{\ms}. With a
maximal interrogation time of $2T = $\sivalue{50}{\ms}, the
interferometer can be operated up to a rate of \sivalue{4}{\Hz}, which
corresponds to a duty cycle of 20\%. With further optimisation of the
2D \ac{mot} system, such as increased cooling beam power, it is
possible to increase the atomic flux and hence increase the experiment
cycling rate. 
\par\noindent
The design of the in-vacuum optical system for the interferometer
light pulses ensures that the fringe visibility is not lost due to
transverse motion of the atoms. Wavefront distortions
of the interferometer light pulses are minimised by not requiring
optical viewports between the Raman laser and the atoms. The beams
are also collimated to a large waist size to reduce dephasing of the
atoms from an intensity gradient. A beam waist of at least
\sivalue{34}{\mm} necessitated an aspheric lens pair in the
collimation optics train. These lenses are not manufactured to the same
optical quality as the triplet lens. It was found that they introduced
an irregular intensity distribution of the laser which led to
dephasing of the atomic states. An improved optical system has been
designed by Shane de Souza, the latest student on this experiment. It
is anticipated that this will improve the wavefront quality for the
interferometer light pulses.
\par\noindent
The sequence of optical and microwave pulses that prepares
atoms in the $\ket{1,0}$ state is very effective at increasing
the population in the $m_F = 0$ state beyond the fraction expected
after an optical molasses. Despite this, there remains a residual
population in the $\ket{1,\pm 1}$ states. These are detected as a
background, which reduces the interferometer fringe contrast. Since
preparing this thesis, the state preparation sequence has been
improved by increasing the polarisation purity of the light driving
the \trans{1}{0} transition. There is now less de-population of the
$\ket{1,0}$ state due to $\pi$-transitions. 
\par\noindent
It has been possible to observe interference with a fringe contrast of
around 5\%. This admittedly low value is likely due to the reasons
stated above. In fact, since reducing the $\ket{1,\pm 1}$ population,
the fringe contrast has increased to around 20\%. Further details of
this can be found in Xiaxi Cheng's PhD thesis~\cite{Cheng2018}.
\par\noindent
The sensitivity to horizontal accelerations was demonstrated by
comparing the interferometer signal in differing levels of vibration
noise. This also highlighted the importance of vibration isolation for
accurate acceleration measurements. When the phase noise is larger
than $2\pi$ \sivalue{}{\radian}, it
is not possible to accurately estimate acceleration using the
interferometer signal. An auxiliary measurement from a classical
sensor is needed to filter the vibration noise. After correcting for
this phase noise, it was found that the stability of the
interferometer signal improved, particularly when the noise was
smaller than $\pi$ \sivalue{}{\radian} - a half-side of a fringe. 
\section{Avenues of Further Research}
Although this work represents the first step towards inertial
navigation using atom interferometry, there is still more to be done.
Further work is planned to extend this system measure accelerations
along 3 axes. This requires a modification of the Raman laser system so
that it can provide enough power from three outputs. This
configuration will need to account for the orientation of the
accelerometer. If the
direction of gravity is not accurately known, it will bias the
acceleration measured by the accelerometer, depending on its
orientation. This is a known problem in inertial navigation. An error
in orientation leads to a
position error that oscillates periodically, known as a Schuler
oscillation~\cite{Schuler1923}. Orienting the accelerometer using a
gyroscope will help to improve the position accuracy, particularly over
durations longer than the Schuler period $2\pi
\sqrt{R_\textnormal{Earth}/g} \approx$
\sivalue{84.4}{\minute}. 
\par\noindent
A comparison of the interferometer's performance to conventional
accelerometers will greatly benefit research aimed towards
practical applications of atom interferometry. This has already been
demonstrated in the context of gravimeters~\cite{Farah2014,Gillot2014}, where a
mobile atomic system has demonstrated greater insensitivity to
vibration noise and better short-term stability than the
state-of-the-art classical counterpart. Additionally, this interferometer has
yet to be tested outside of a laboratory environment. A recent
experiment has investigated the bias stability of a hybrid sensor in a
simulated harsh environment~\cite{Cheiney2018}. An extension of this
work to measuring a range of accelerations would further
support the viability of cold atom inertial sensors.  
\section{Concluding Remarks}
The application of newly understood physics has often led to the
development of more advanced technology. In turn, this has contributed
to further progress in scientific disciplines, and society in general.
This philosophy can be applied to the application of matter-wave
interferometry to inertial sensing. There already exists a significant
body of research that demonstrates the technical feasibility of
measuring inertial forces using cold atomic systems. In addition, the
practical requirements for navigation have served to identify a need
for the unique benefits of atom interferometers. This work presented
in this thesis has helped to address this need. It is anticipated that
the application of quantum mechanics to technology will
bring practical benefits as well as enable a deeper understanding of
physical phenomena.
