\documentclass{article}
\usepackage{siunitx}
\usepackage{amsmath}
\newcommand{\sivalue}[2]{\SI[mode=text]{#1}{#2}}
\title{An Atom Interferometer for Measuring Horizontal Accelerations}
\date{}
\begin{document}
\maketitle
Matter-wave interferometry has enabled high precision measurements of
    inertial forces such as gravity and the
    Coriolis force. This is facilitated by the long-term stability of
    the physical properties of atoms and lasers. Recent experiments have
    demonstrated the operation of portable, robust sensors using atom
    interferometry. This has potential uses in the context of inertial
    navigation, where conventional devices suffer from long-term
    drifts due to bias instability. Furthermore, determining position
    via dead reckoning requires minimisation of dead time between
    measurements. This thesis presents the development of an atom
    interferometer for measuring horizontal accelerations. In this
    configuration, gravity induces motion across
    the laser wavefront, which constrains the tolerable level of
    wavefront distortions. Effective control of the experiment allows
    the interferometer to be operated at a rate of \sivalue{4}{\Hz}. A cold ensemble of $10^6$ atoms in the same
    internal state is prepared in \sivalue{150}{\ms}. The
    interferometer operates
    using a sequence of three laser pulses separated by
    $T=$ \sivalue{25}{\ms} to achieve sensitivity
    to horizontal accelerations. Combining this with a classical
    accelerometer provides a method of correcting for
    vibration-induced noise, as well as determining the interferometer
    fringe order. After an integration time of \sivalue{70}{\s}, the
    sensitivity to horizontal accelerations is better than
    \sivalue{1e-6}{\m\s\tothe{-2}}. Effects which limit this
    sensitivity are discussed.
  \end{document} 
