\titlepage[Department of Physics\\
Centre for Cold Matter]{A thesis submitted to Imperial
  College London for partial
fulfillment of the requirements for the degree of Doctor of Philosophy }
\begin{abstract}
    Matter-wave interferometry has enabled high precision measurements of
    inertial forces such as gravity and the
    Coriolis force. This is facilitated by the long-term stability of
    the physical properties of atoms and lasers. Recent experiments have
    demonstrated the operation of portable, robust sensors using atom
    interferometry. This has potential uses in the context of inertial
    navigation, where conventional devices suffer from long-term
    drifts due to bias instability. Furthermore, determining position
    via dead reckoning requires minimisation of dead time between
    measurements. This thesis presents the development of an atom
    interferometer for measuring horizontal accelerations. In this
    configuration, gravity induces motion across
    the laser wavefront, which constrains the tolerable level of
    wavefront distortions. Effective control of the experiment allows
    the interferometer to be operated at a rate of \sivalue{4}{\Hz}. A cold ensemble of $10^6$ atoms in the same
    internal state is prepared in \sivalue{150}{\ms}. The
    interferometer operates
    using a sequence of three laser pulses separated by
    $T=$ \sivalue{25}{\ms} to achieve sensitivity
    to horizontal accelerations. Combining this with a classical
    accelerometer provides a method of correcting for
    vibration-induced noise, as well as determining the interferometer
    fringe order. After an integration time of \sivalue{70}{\s}, the
    sensitivity to horizontal accelerations is better than
    \sivalue{1e-6}{\m\s\tothe{-2}}. Effects which limit this
    sensitivity are discussed.
 \end{abstract}
\begin{declaration}
  I declare that this thesis is the result of my own work. All sources
  used for this work have been clearly referenced in accordance with
  the departmental requirements.   
        \vspace*{1cm}
        \begin{flushright}
        Jimmy Stammers
        \end{flushright}
        \vfill
  The copyright of this thesis rests with the author and is made
available under a Creative Commons Attribution Non-Commercial No
Derivatives licence. Researchers are free to copy, distribute or
transmit the thesis on the condition that they attribute it, that they do
not use it for commercial purposes and that they do not alter,
transform or build upon it. For any reuse or redistribution,
researchers must make clear to others the licence terms of this
work.
\end{declaration}
\begin{acknowledgements}
\noindent
There are many people whose support I wish to acknowledge. Firstly, I would like to thank Ed Hinds for his
supervision and guidance throughout the project. His attention to
detail and desire to understand the experiment from first principles
have taught me a lot about how to approach scientific research. I also
owe a great deal of thanks to the various post-docs I have worked
with. Yu-Hung Lien and Giovanni Barontini were both immensely helpful when I was a new student
and unfamiliar with the techniques required to trap and cool Rubidium
was invaluable. After they moved on, we were soon joined by Indranil
Dutta. As someone who had just come from a well-established
atom interferometry experiment, he was a welcome addition
to our project. I owe him a great deal for everything he taught me
about how to build an interferometer and to diagnose the inevitable
problems along the way. Later on, we were joined by Joe Cotter and
Teodor Krastev, who were both very helpful in developing the
interferometer into a more robust device. Joe was. Theo's expert
Then there are the other students on the Navigator project. It was a
great pleasure working with Xiaxi Cheng as he was always willing to
work on understanding the problem at hand. The next stage of the
project is being undertaken by Shane de Souza and Nicola Bilton. I
have enjoyed working with the both of them and being able to pass on
some of the knowledge that I have learned to them. I wish them all the
best for the continuation of their PhDs. 
\par\noindent 
I have thoroughly enjoyed working as part of CCM. The friendly,
collaborative environment, as well as the social cohesion of the group
is something often hard to come by. I would like to particularly
acknowledge Kyle Jarvis and Dylan Sabulsky. The three of us started
our PhDs together, and I feel that we have a lot of shared experiences
of life as PhD students, both positive and negative. I also owe a
great deal of thanks to Sanja Maricic and Jon Dyne. Without Sanja's
ability to deal with the complex and intricate details of Imperial
College's administration and Jon's expertise at turning
any Inventor design into a precision-engineered component, I am
certain my PhD would have gone much differently.
\par\noindent
Finally, I would like to acknowledge people within my personal life
who have helped me get to where I am today. My parents, Si\^{a}n and
Carl have been very supportive of my choices in life. I am grateful
for everything they have given me. Last, but by no means least, I would
like to thank Vickie for our time together. She has encouraged me to pursue the things I
enjoy and to focus on the positive aspects of work, as well as life. Even when we were thousands of miles apart, she continued to
offer her kind words of encouragement and I will forever be grateful
for her unique support.
\end{acknowledgements}
%\begin{preface}
%    This thesis describes my research on various aspects of\ldots
%\end{preface}

%\begin{frontmatter}
    %   \dedication{For the World}
    %   ...
%\end{frontmatter}

\tableofcontents
\listoffigures
\listoftables
\frontquote{I may not have gone where I intended to go, but I think I
have ended up where I needed to be}{Douglas Adams}
