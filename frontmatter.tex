\titlepage[Department of Physics]{A thesis submitted to Imperial
  College London for partial
fulfillment of the requirements for the degree of Doctor of Philosophy }
\begin{abstract}
    Matter-wave interferometry has enabled high precision measurements of
    inertial forces such as gravity and the
    Coriolis force. This is facilitated by the long-term stability of
    the physical properties of atoms and lasers. Recent experiments have
    demonstrated the operation of portable, robust sensors using atom
    interferometry. This has potential uses in the context of inertial
    navigation, where conventional devices suffer from long-term
    drifts due to bias instability. Furthermore, determining position
    via dead reckoning requires minimisation of dead time between
    measurements. This thesis presents the development of an atom
    interferometer for measuring horizontal accelerations. In this
    configuration, gravity induces motion across
    the laser wavefront, which constrains the tolerable level of
    wavefront distortions. Effective control of the experiment allows
    the interferometer to be operated at a rate of \sivalue{4}{\Hz}. A cold ensemble of $10^6$ atoms in the same
    internal state is prepared in \sivalue{150}{\ms}. The
    interferometer operates
    using a sequence of three laser pulses separated by
    $T=$ \sivalue{25}{\ms} to achieve sensitivity
    to horizontal accelerations. Combining this with a classical
    accelerometer provides a method of correcting for
    vibration-induced noise, as well as determining the interferometer
    fringe order. After an integration time of \sivalue{70}{\s}, the
    sensitivity to horizontal accelerations is better than
    \sivalue{1e-6}{\m\s\tothe{-2}}. Effects which limit this
    sensitivity are discussed.
 \end{abstract}
\begin{declaration}
  I declare that this thesis is the result of my own work. All sources
  used for this work have been clearly referenced in accordance with
  the departmental requirements.   
        \vspace*{1cm}
        \begin{flushright}
        Jimmy Stammers
        \end{flushright}
        \vfill
  The copyright of this thesis rests with the author and is made
available under a Creative Commons Attribution Non-Commercial No
Derivatives licence. Researchers are free to copy, distribute or
transmit the thesis on the condition that they attribute it, that they do
not use it for commercial purposes and that they do not alter,
transform or build upon it. For any reuse or redistribution,
researchers must make clear to others the licence terms of this
work.
\end{declaration}
\begin{acknowledgements}
    There are many people to whom I owe a great deal of thanks. 
\end{acknowledgements}

\begin{preface}
    This thesis describes my research on various aspects of\ldots
\end{preface}

%\begin{frontmatter}
    %   \dedication{For the World}
    %   ...
%\end{frontmatter}

\tableofcontents
\listoffigures
\listoftables
\frontquote{I may not have gone where I intended to go, but I think I
have ended up where I needed to be}{Douglas Adams}
