\chapter{Acceleration-Sensitive Interference}\label{chap:atom_int}

\section{Chapter Outline}
This chapter describes the aspects of the project aimed at observing
matter-wave interference in \ac{rb87} and its subsequent
characterisation. The laser system used to drive the necessary Raman
transitions is presented in~\SectionRef{sec:msquared_laser}. This is
followed by a discussion of the methods used to detect the population
in each internal state in~\SectionRef{sec:atom_detection}. The Raman
transition spectrum and
dynamics of the atoms during each Raman pulse are discussed
in~\SectionRef{sec:atomint_rabiosc}. This chapter continues
with an overview of indentified sources of noise and their impact on
the interferometer's sensitivity to accelerations
in~\SectionRef{sec:atomint_sensitivity}. Finally, a presentation of observed
interference and an analysis of its sensitivity to accelerations is
given in~\SectionRef{sec:atomint_accelerations}.
\section{The M-Squared Laser System}\label{sec:msquared_laser} 
  This section describes the laser system manufactured by \textit{M-Squared
  Lasers}, which is used to drive Raman transitions. 
  A \ac{pll} controls the phase difference of two Ti:Sapphire lasers
  by comparing the beat note with a stable local oscillator\nocite{Lautier2014b}\nocite{Marino2008}. An overview of the laser
  system can be found in~\SectionRef{subsec:msquared_overview}, which includes
  the techniques used to externally communicate with the laser's ICE-BLOC
  control modules. The control of the frequency and phase-lock is then described
  in~\SectionRef{subsec:msquared_control}. Finally, this section concludes
  in~\SectionRef{subsec:dcs_module} with a description of the DCS
  module which is used to control
  the amplitude, frequency and phase of the Raman laser beat-note
  during the experiment.
\subsection{Laser System Overview}
\label{subsec:msquared_overview} The Raman laser system contains two
SolsTiS lasers, each generating \sivalue{780}{\nm} light by pumping a Ti:Sapphire crystal housed inside a resonator.
The output light is frequency-stabilised using piezo-electric stacks to adjust
the resonator length~\cite{Drever1983}. A schematic diagram of this laser system is given
in~\FigureRef{fig:msquared_laser}. Each laser is pumped using a
\sivalue{12}{\W} \textit{Lighthouse Photonics} Sprout laser at
\sivalue{532}{\nm}. One SolsTiS acts as the master frequency locked
to an absorption feature in the saturated absorption spectrum of
\ac{rb87}. The second is slaved to this using a phase-locked loop to keep their
beat frequency constant. The two beams are mixed on a
\ac{pbs}, so that they are orthogonally polarised. Two \acp{aom} control the
output power. 
%A planned upgrade for this system will have multiple output ports
%for the Raman light, which will require independent control. 
\par\noindent 
The system contains 4 ICE-BLOC modules which implement various types of control. The
first two (one for each Solstis) are used to stabilise the output power of each
laser by feeding back to the corresponding Sprout laser. They are also used to
coarsely adjust the output frequency, which is measured using a
\textit{HighFinesse} wavemeter. The third is used for the \ac{pll}
and feeds-back onto the slave laser to control both the frequency and phase of
the optical beat-note between the two lasers. The final ICE-BLOC, referred to
as the DCS module, is used to control the lasers in real-time during the
experiment.  
\begin{figure}
	\centering \fontsize{10pt}{10pt}
	\resizebox{0.8\textwidth}{!}{\input{msquared_laser.pdf_tex}}
	\caption[M-Squared Laser System Schematic]{Schematic Diagram of the M-Squared
		laser system. Two SolsTiS lasers provide the two Raman frequencies, which
		are fibre coupled onto the orthogonal axes of a \ac{pm} fibre. Control of
		the power, frequency and phase as required to drive and control
    the Raman transitions is
		handled by the four ICE-BLOC modules indicated in blue. Further detail of
		this control is given in the text.} \label{fig:msquared_laser}
\end{figure}

\subsubsection{External ICE-BLOC Control}
The ICE-BLOC modules are able to communicate with each other using an Ethernet
hub. Another computer connected to this network is able to control them by
accessing a web page that each module hosts. These web pages control the ICE-BLOCs
by sending structured JSON messages. This graphical interface can be bypassed by
directly communicating these messages. This is done using MOTMaster so that
various parameters, such as the frequency and phase of the Raman beat-note, can
be automatically varied between experiment cycles. 
\subsection{Frequency and Phase Control}\label{subsec:msquared_control}

\subsubsection{Master Lock}
The frequency of the master laser is stabilised using saturated absorption
spectroscopy in a Rubidium vapour cell. Part of the beam is picked off and
modulated by an \ac{eom}. The positive frequency sideband is used to lock the
master laser to the 2,3 crossover feature. In effect, this means that the
modulation frequency of the \ac{eom} sets the one-photon detuning of the Raman
transition. The modulation frequency is set so that the master laser frequency
is \sivalue{1.13}{\giga\hertz} below the \trans{2}{3} transition. This
frequency is chosen because the light shift of the clock transition
vanishes there provided the two Raman beams have the right intensity
ratio as discussed in~\SectionRef{subsec:light_shift}. This ensures
that the resonant frequency is independent of variations in the
over-all intensity of the Raman beams.
\subsubsection{Frequency and Phase Lock} The optical beat-note between the two
lasers is measured using a fast photodiode. The signal from this is used in a
\ac{pll} to fix the relative phase between the two lasers. A frequency
divider halves the frequency of the signal before comparing it to a \ac{vco} of
around \sivalue{3.4}{\giga\hertz}. This creates an error signal which used to
control both the frequency and phase of the beat-note by feeding back to the
slave laser Solstis. The relative phase between the two lasers is adjusted using
an analogue phase shifter and the frequency difference is controlled by tuning
the \ac{vco} frequency. \par\noindent The beat-frequency of the Raman lasers can
be chirped by triggering a ramp of the control voltage to the \ac{vco}. For
chirp rates of lower than \sivalue{24}{\mega\hertz\per\second}, the phase-lock
is able to keep the beat-note phase-coherent during the chirp.  
\subsection{The DCS Module}\label{subsec:dcs_module} 
The DCS module is used to control the
output of the lasers during the experiment. It uses an on-board
\ac{dds} to
synthesise the \sivalue{80}{\mega\hertz} driving frequencies for each \ac{aom}.
The majority of the control is done using an \ac{fpga} that synthesises a timed
sequence of analogue and digital voltage waveforms. An example of a sequence
created using the DCS web interface is shown in~\FigureRef{fig:dcs_module}. The
sequence is segmented into individual steps and each channel can be separately
configured, much like the MOTMaster user interface. \par\noindent This module is
used to control the amplitude, frequency and phase of each Raman pulse. The
pulse amplitude is shaped using an analogue voltage to control the power of the
RF frequency. The voltage output has been calibrated so that the pulse can be shaped to
produce a square, Gaussian or Blackman amplitude envelope. A frequency
chirp of the beat-note is
optionally triggered by sending a digital pulse to the \ac{pll} ICE-BLOC.
\par\noindent The synthesiser can be configured to run continuously, or to wait
at a chosen timestep for an external trigger. It can also iterate through a set
number of parameters, such as timestep duration or phase shift by re-building
the sequence after each cycle.  
\begin{sidewaysfigure}[!htbp] 
  \centering
	\resizebox{1\textwidth}{!}{\includegraphics{dcs_module.pdf}}
	\caption[DCS Module User interface]{DCS module user interface. The sequence is
		synthesised from individual steps. The parameters of each Raman laser pulse
		can be configured independently.} 
  \label{fig:dcs_module} 
\end{sidewaysfigure}
\section{Atom Detection}\label{sec:atom_detection} 
This section describes the methods used to
measure the number of atoms in each hyperfine ground state and infer
the interferometer phase. It begins with a
presentation the optical setup used to collect fluorescent light on a
photodiode in~\SectionRef{subsec:optical_setup}.
The scheme used to detect the atoms by
driving \(\sigma^+\) transitions is then described in~\SectionRef{subsec:optical_setup}.
This concludes with a
discussion on converting the measured photodiode signals into atom
number and interferometer phase
in~\SectionRef{subsec:phase_measurement}.
\subsection{Optical Setup}\label{subsec:optical_setup}
Out aim in this device is to reach the standard quantum noise limit,
which comes from the quantum projection noise. Expressed as a
fractional accuracy this is given approximately by
$N^{-1/2}$~\cite{Bollinger1996}, where $N$ is the number of atoms detected.
The CCD used initially was not sensitive enough for this as there wass a
significant amount of noise in reading out the charge collected at each pixel.
Instead, a more sensitive photodiode is used to detect the atoms. With a
suitably high bandwidth, the readout time is much faster than the CCD as well,
so that the atoms can be detected well before they fall out of the field of
view. \par\noindent A diagram of the setup used to detect the atoms is given
in~\FigureRef{fig:photodiode_optics}. It is a triplet system which uses
lenses with focal lengths \sivalue{150}{\milli\metre},
\sivalue{75}{\milli\metre} and \sivalue{60}{\milli\metre}, with the
\sivalue{150}{\mm} lens closest to the atoms and the \sivalue{60}{\mm}
lens closest to the photodiode. A ray-tracing simulation of the optical
system indicates spherical aberrations on the image. This is caused by
the third lens, which was added to shorten the back focal length. The
front lens has a diameter of
\sivalue{50.4}{\milli\metre}, so the solid angle subtended by the
optics is \(4\pi \times\)\num{7.1e-3}\si{\steradian}.
\begin{figure}[!htbp] 
  \centering \fontsize{24pt}{24pt}
	\resizebox{0.7\textwidth}{!}{\input{photodiode_optics.pdf_tex}}
	\caption[Optical setup for Photodiode Detection]{Optical setup for photodiode
		detection. A triplet lens system focuses light from radiated from the atoms
		onto a photodiode. This is mounted using a translation stage to
  position the photodiode at the back focal point.}
  \label{fig:photodiode_optics}
\end{figure}

\subsubsection{Photodiode Calibration}
The photodiode used is a \textit{Femto LCA-S-400K-SI},
which has a trans-impedance amplifier with a bandwidth of \sivalue{400}{\kilo\hertz} and a photo-sensitive area
with a diameter of \sivalue{3}{\milli\metre}. The scaling factor from
incident optical power to output voltage was measured as \sivalue{1.84e6}{\volt\per\watt}. 
%\begin{figure}[htpb]
%  \centering
%  \includegraphics[width=0.8\linewidth]{Figures/Chapter6/pd_calib.pdf}
%  \caption{Name}
%  \label{fig:name}
%\end{figure}
\subsection{Detection using \(\sigma^+\) transitions}\label{subsec:photodiode_setup}
The atoms are detected using resonance fluorescence from the two
vertically aligned \ac{mot} beams in the presence of a vertical
magnetic field. For the \ac{mot} and molasses these are polarised
$\sigma^+$ and $\sigma^-$, but for this detection step, we use a
liquid-crystal \ac{hwp} to give both beams $\sigma^+$ polarisation.
This causes the atoms to be optically pumped into \(\ket{2,2}\) and cycle on the
\(\ket{2,2} \rightarrow \ket{3,3}\) transition, which allows the atoms
to scatter many photons with minimum probability of unwanted optical
pumping into $\ket{F=1}$.
\par\noindent
\FigureRef{fig:detection_scheme} shows the setup used to invert the
polarisation of one \ac{mot} beam prior to detection. The liquid-crystal
waveplate is an electro-optical device whose birefringence changes
when an ac voltage is applied across it. The waveplate is placed at
the output of the downward-propagating (\(\vec{z}_-\))
collimator. The liquid-crystal waveplate is triggered to rotate the
incoming linearly polarised light by $\pi/2$ \sivalue{}{\radian}. 
\begin{figure}[!htpb]
    \centering
    \fontsize{14pt}{14pt}
    \resizebox{0.5\textwidth}{!}{\input{detection_scheme.pdf_tex}}
    \caption[Scheme to invert beam polarisation.]{Scheme to invert
      beam polarisation. In the \ac{mot} loading phase of the
      experiment, the liquid crystal \ac{hwp} is oriented to give a
      right-hand circular polarised beam shown in blue. Prior to
      detection, a digital pulse triggers a re-orientation of its slow
      axis. This results in a left-hand circular polarised beam, shown
      in red.}\label{fig:detection_scheme}
\end{figure}
\subsubsection{Detection Sequence}\label{subsec:detection_sequence}
The sequence used to detect the atoms is shown
in~\FigureRef{fig:detection}. Shortly before the sequence starts, the
bias field is aligned to the \(\vec{z}\) axis and the liquid-crystal
waveplate is triggered to change the handedness of the \(\vec{z}_-\)
beam. The cooling laser frequency is set so it is detuned by
\(\delta_D =\) \sivalue{3}{\mega\hertz}
below the \trans{2}{3} transition and the repump laser is set to
resonance with the \trans{1}{2} transition. This creates an optical
molasses which avoids heating the atoms so that they remain in the
detection volume for a longer period of time. The intensity of the
light is reduced to around \(3 I_\text{sat}\). As shown below, this
intensity was empirically found to minimise the variance in output
voltage. The acquisition of the photodiode voltage is
triggered to start at the first Dwell time. Durin The cooling light is first
switched on without any repump light so that only atoms in \(\ket{F=2}\) scatter light. After this, the repump
is switched on, so that atoms in \(\ket{F=1}\) are optically pumped
into \(\ket{F=2}\) and all the atoms scatter light. Now the
fluorescence measures the total number of atoms $N$. This repump light is a
sideband of the cooling laser, so the total output is increased
to ensure that the intensity of the cooling light remains constant.
Each detection step lasts \sivalue{250}{\micro\second}, but the
first \sivalue{50}{\micro\second} is discarded to allow time for the
intensity to stabilise and for optical pumping into \(\ket{F=2}\).
All the atoms are then blown away by switching off one of the detection
beams before the sequence is repeated to collect a
background signal.
\begin{figure}[!htbp] 
  \centering
  \fontsize{14pt}{14pt}
  \resizebox{0.8\textwidth}{!}{\input{detection.pdf_tex}} 
  \caption[State detection sequence timing]{Timing diagram for state
    detection. Atoms in \(\ket{F=2}\) are detected first, then repump
    light pumps the \(\ket{F=1}\) atoms into $\ket{F=2}$ so they are detected as well. A
background light measurement follows the measurement of the atom
numbers.}
	\label{fig:detection} 
\end{figure}
\subsubsection{Maximum Detection Time}\label{subsubsec:intensity_dependendce}
As the atoms scatter light during detection, the cloud will be heated
and expand due to the momentum exchanged from absorption and
spontaneous emission. The atoms are only cooled along the axis of the
detection beams, so the heating rate is greatest along the other two
axes. It is necessary to ensure that the
heating rate is low enough that the atoms remain within the
detection beam for the entire detection time. A requirement on the
maximum detection time can be obtained as follows. The momentum of
an atom scattering photons follows a random walk, so 
if the cloud has a
Gaussian spatial distribution with an
initial width of \(\sigma_0\), the width at a later time of
is given by
\begin{equation}
\sigma_x^2(t) = \sigma_0^2 + \frac{2 n_p v_r^2 t^2}{3} 
  \label{eq:width_scattering}
\end{equation}
where \(v_r = \frac{\hbar k}{m_\text{rb}} =
\) \sivalue{6}{\milli\metre\per\second} is the recoil velocity and \(n_p\) is the
number of photons scattered. The factor of $2/3$ is because only the
transverse component of the recoil is relevant here. To remain within the detection region,
the width of the cloud must be smaller than the detection beam waist
\(w\), so the detection time must satisfy
\begin{equation}
  t_D \ll \sqrt{\frac{3 \left(w^2-\sigma_0^2\right)}{2 \left(n
  v_r^2\right)}}
  \label{eq:detection_time}
\end{equation}
For a beam waist of \sivalue{7.5}{\milli\metre}, initial cloud size of
\sivalue{5}{\milli\metre} and a maximum scattering rate of
\sivalue{2e7}{\per\second} the detection time must be much less than
\sivalue{4.7}{\milli\second}. This inequality is amply satisfied by
our detection time of \sivalue{100}{\micro\s}.
%\begin{equation}
%  p(x,t) = \int \frac{1}{\sqrt{2\pi} e^{-\frac{(x + \frac{p}{m}
%    t)^2}}{2 \sigma_x^2} \frac{1}{\sqrt{2\pi D t} e^{-\frac{p^2}{2 D
%      t}} \mathrm{d}t
%  \label{eq:prob_atom_diff}
%\end{equation}
\subsubsection{Detection Intensity}
The intensity for detection was chosen by varying the total power
in the detection beams and recording the photodiode voltage for a
fixed detection time of \sivalue{200}{\micro\second}.
\FigureRef{fig:photodiode_intensity_calib} shows the average voltage
measured when detecting atoms in the \(\ket{F=2}\) state as the
intensity of the light increases. The saturation parameter \(s\) is
inferred from the control voltage used control the light
through the \ac{aom} at the output of the
\Muquans laser (see~\FigureRef{fig:muquans_cooling}) and the peak
intensity of the \ac{mot} beams. By the time the atoms are detected,
they have moved away from the region of peak intensity, so the
intensity on the atoms is smaller than the applied control voltage
would indicate if they were at the peak intensity. A fit parameter $b$
is introduced to account for this. A non-linear least squares fit
to the function
\begin{equation}
  v = a\frac{b s}{1 + b s + 4 \left(\delta_D/\Gamma\right)^2}
  \label{eq:voltage_fit}
\end{equation}
gives a scaling for the intensity of \(b = 0.83\). The variance in the
measured voltage (for a constant mean number of atoms) is minimised when the
intensity is around 3\(I_\text{sat}\). Above this intensity, there is
a significant depopulation into \(\ket{F=1}\) caused by off-resonant
excitations to the \(\ket{F'=2}\) state.
This is evident in the voltage signal over time, which is
shown in \FigureRef{fig:detection_time} for various intensities. At an
intensity of 3\(I_\text{sat}\), around 5\% of the population is
pumped out of \(\ket{F=2}\).
\begin{figure}[htpb!]
  \centering
  \includegraphics[width=0.7\textwidth]{photodiode_intensity}
  \caption[Photodiode output voltage for increasing detection beam
  intensity. ]{Photodiode output voltage for increasing detection beam
  intensity. The red dashed line indicates a fit
to~\EquationRef{eq:voltage_fit} to estimate the scaling factor for the
saturation parameter \(s\).}
  \label{fig:photodiode_intensity_calib}
\end{figure}

\begin{figure}[htpb!]
  \centering
  \includegraphics[width=0.7\textwidth]{photodiode_voltage_time}
  \caption[Photodiode voltage for varying detection times.]{Photodiode voltage over time during detection for \(s =
    0.5, 1, 3, 7, 10\) in order from purple to yellow.}
  \label{fig:detection_time}
\end{figure}

\subsection{Measuring the Occupation Probability}\label{subsec:phase_measurement}

The occupation probability of the $\ket{F=2}$ state is obtained by measuring the proportion of
atoms in each hyperfine ground state. The voltage of the amplified
photodiode signal is related to the number of atoms
\(n_\text{at}\) that scatter light on the cycling transition by
\begin{equation}
 V &= \eta R_\text{sc}(s,\delta) n_\text{at} \hbar \omega G 
  \label{eq:pd_signal}
\end{equation}
where \(\eta = \Omega/4\pi\) is the fractional solid angle subtended by the
collection optics, \(\hbar\omega = \sivalue{1.6}{\electronvolt}\) is
the photon energy, \(R_\text{sc}\) is the scattering rate per atom defined
in~\EquationRef{eq:scattering_rate} and \(G\) is the photodiode
conversion gain of \sivalue{1.84e6}{\V\per\W}. At the saturation intensity and a detuning of
\sivalue{3}{\mega\hertz}, the voltage measured per
atom is around \sivalue{30}{\nano\volt} per atom. With a detection
time of \sivalue{200}{\micro\s}, around 7 photons per atom are
detected. The probability of
an atom occupying \(\ket{F=2}\) is estimated as follows
\begin{equation}
  \text{P}_{\ket{F=2}} =
  \frac{\text{N}_2}{\text{N}_\text{Tot}}
  \label{eq:prob_measurement}
\end{equation}
Here the subscript 2 indicates the average voltage measured with
cooling light only (detecting $\ket{F=2}$ atoms) while the subscript
Tot indicates the average voltage with the repump light as well
(detecting all the atoms). 
The interferometer phase difference \(\Phi\) is determined from \EquationRef{eq:prob_measurement} using
\begin{equation}
  \text{P}_{\ket{F=2}} = \text{P}_0 - \frac{C}{2}\cos(\Phi)
  \label{eq:interferometer_phase}
\end{equation}
where P\(_0\) is the mean probability of detecting atoms in
\(\ket{F=2}\) and \(C\) is the interferometer fringe contrast. These
are experimentally determined by varying $\Phi$ as described
in~\SectionRef{sec:fringe_cal}.

\subsubsection{Atom Number Bias}\label{subsec:atom_number_bias}
After the initial state preparation is complete, there are still some
residual atoms left in the states $\ket{F=1, m_F = \pm 1}$. Also, as
shown in~\FigureRef{fig:detection_time}, the photodiode signal that
yields $N_2$ is not constant because some of the $\ket{F=2}$ atoms are
optically pumped into the $\ket{F=1}$ state. In the following, we
consider how both these defects affect the validity
of~\EquationRef{eq:interferometer_phase}.
\par\noindent
If atoms are pumped out of
\(\ket{F=2}\) at a rate \(\gamma\), then the number of atoms in the
number of atoms in $\ket{F=2}$ is given by
\begin{equation}
  n_2(t) = n_{20}e^{-\gamma t}
  \label{eq:n2_time}
\end{equation}
where \(n_{20}\) is the initial number in \(\ket{F=2}\). After
averaging over a time \(\tau\), this gives
\begin{equation}
  N_2 = \frac{n_{20} (1-e^{-\gamma \tau})}{\gamma \tau}
  \label{eq:n2_avg}
\end{equation}
The total number of atoms, measured during the second detection pulse
is
\begin{equation}
  N_\textnormal{Tot} =  n_{10} + n_{20} + n_{\pm 1}
  \label{eq:n1_avg}
\end{equation}
where \(n_{\pm 1}\) is the background population in
\(\ket{1,\pm{1}}\) and $n_{10}$ is the initial population in
$\ket{1,0}$. The detected probability is then
\begin{align}
  \text{P} &= \frac{N_2}{N_\text{Tot}} \nonumber \\
    &= \frac{\frac{n_{20} (1-e^{-\gamma \tau})}{\gamma \tau}}{n_{10} +
    n_{20} + n_{\pm 1}}
    \label{eq:prob_bias}
\end{align}
which is not the same as occupation probability from the
interferometer fringe
\begin{equation}
  \text{P}_0 = \frac{n_{20}}{n_{10} + n_{20}}
\end{equation}
Dividing~\EquationRef{eq:prob_bias} by $n_{10}+n_{20}$, the detected
probability is expressed in terms of $\text{P}_0$ as
follows 
\begin{equation}
  \text{P} = \frac{\text{P}_0}{1+\epsilon}\frac{1-e^{-\gamma \tau}}{\gamma \tau}
\end{equation}
where \(\epsilon = \frac{n_{\pm{1}}}{n_1+n_2}\) is the ratio of the
number of residual atoms to the number in the interferometer.
Similarly, the contrast $C$ of the detected fringe is related to the
actual contrast $C_0$ by 
\begin{equation}
  C = \frac{C_0}{1+\epsilon}\frac{1-e^{-\gamma \tau}}{\gamma \tau}
  \label{eq:contrast}
\end{equation}
For small $\gamma \tau$, this multiplicative factor is
well-approximated by $\frac{1-\gamma \tau}{1+\epsilon}$. 
From~\FigureRef{fig:detection_time}, around 7\% of the signal is lost
for \(s = 3\), so $\gamma = $ \sivalue{360}{\per\s} and $\tau =$
\sivalue{200}{\micro\s}. From the imperfect state preparation, after
velocity selection there is roughly equal population in the $m_F = 0$
state as the $m_F = \pm 1$ states, so we assume $\epsilon = 1$. 
Now the detected contrast is $C = 0.46 C_0$, which includes a factor
of $1/2$ from the $m_F = \pm 1$ atoms. This background population
greatly reduces the detected fringe contrast. Increasing the purity of
the state prior to interferometry will certainly be required to
improve this fringe contrast. 
\section{Individual Pulse Characterisation} \label{sec:atomint_rabiosc}
This section presents a characterisation of the pulses used to drive
Raman transitions between the two hyperfine ground states. First, the
properties of the Raman transition spectrum are presented
in~\SectionRef{subsec:raman_spec}. Following this, a discussion of
cancelling the systematic phase from a differential ac Stark Shift is
given in~\SectionRef{subsec:light_shift}. Finally, this section
concludes with specific details about the individual pulses used in
the experiment. The first Raman pulse, which is used to select a
subset of atoms with a narrow velocity spread, is presented
in~\SectionRef{subsec:vel_select}. This section concludes with a presentation
of the dynamics of the three pulses used to coherently control the
atoms during the interferometer in~\SectionRef{subsec:int_pulses}
\subsection{Raman Transition Spectrum}\label{subsec:raman_spec}
When a circularly polarised light beam excites an atom, the angular
momentum of the light is transferred to the atom. Therefore, a
$\sigma^+$ circularly polarised beam raises the $m_F$ quantum number
by 1 when a photon is absorbed. Similarly, the stimulated emission
induced by a $\sigma^+$ beam lowers $m_F$ by 1. In the case of our
Raman transitions, one beam (with wavevector $\textbf{k}_1$) excites
the atom and another (wavevector $\textbf{k}_2$) stimulates it back
down to the other hyperfine ground state.
\TableRef{tab:raman_polarisation} shows the polarisation selection
rules for these transitions.
\par\noindent
\FigureRef{fig:raman_spectrum} shows an example of the Raman
transition spectrum taken with atoms that have been launched along the
Raman axis at about \sivalue{8}{\cm\per\s}. The beat frequency
between the two Raman lasers is scanned and the light is pulsed for
\sivalue{160}{\micro\second} to drive atoms from $\ket{1,0}$ into the \(\ket{F=2}\)
state. The labels in~\FigureRef{fig:raman_spectrum} indicate the
initial and final $m_F$ values. There is a large narrow peak labelled
$0\rightarrow0$ close to the hyperfine splitting
frequency. This is a result of velocity-insensitive co-propagating
transitions\footnote{When the two light fields are co-propagating, the
  Doppler shift, being proportional to \(\vec{k}_1 -
  \vec{k}_2\) is close to zero}. Looking at the entry
  in~\TableRef{tab:raman_polarisation} for $\Delta m = 0$, this peak
  indicates that the co-propagating beams do not have exactly the
  $\sigma^+ - \sigma^+$ polarisation that was intended. This is further supported by the fact
that there are \(\Delta m = \pm 1\) transitions, which can only occur
if one of the lasers drives a \(\pi\) transition. The Zeeman shifts on
the co-propagating transitions between \(\ket{F=1,0}
\rightarrow \ket{F=2,1}\) and \(\ket{F=1,1}\rightarrow \ket{F=2,1}\)
are \sivalue{95}{\kilo\hertz} and \sivalue{189.5}{\kilo\hertz}, and
are perfectly consistent with the known applied magnetic field of \sivalue{1.4}{\gauss}.
%\begin{table}
%  \centering
%  \begin{tabular}{ccccc}
%    \toprule
%    & & \multicolumn{3}{c}{\(\vec{k}_2\)} \\
%     \midrule
%     & & \(\sigma^-\) & \(\pi\) & \(\sigma^+\)\\
%     \multirow{3}{*}{\(\vec{k}_1\)} & \(\sigma^-\) &  c\(_1\) &
%     c\(_2\)&
%     --  \\
%     & \(\pi\) &c\(_3\) & -- & c\(_4\) \\
%     & \(\sigma^+\) & --& c\(_5\)& c\(_6\)\\
%    \bottomrule
%  \end{tabular}
%  \caption[Raman transition polarisation configurations]{Labels for Raman transitions excited
%    from \(\ket{F=1}\) by \(\vec{k}_1\) and stimulated into
%  \(\ket{F=2}\) by \(\vec{k}_2\).}
%  \label{tab:raman_polarisation}
%\end{table}
\begin{table}
  \centering
  \begin{tabular}{ccccccc}
    \toprule
     & & \multicolumn{5}{c}{\(\ket{F=2,m}\)} \\
     \midrule
     & & -2 & -1 & 0 & 1 & 2 \\
     \multirow{3}{*}{\(\ket{F=1,m}\)} & -1 & (c\(_2\),c\(_4\)) &
     (c\(_1\),c\(_6\)) &(c\(_3\),c\(_6\))& -- & --  \\
     & 0 &-- & (c\(_2\),c\(_4\))& (c\(_1\),c\(_6\)) & (c\(_3\),c\(_6\))
     &-- \\
     & 1 & --& --&(c\(_2\),c\(_4\))& (c\(_1\),c\(_6\)) & (c\(_3\),c\(_6\)) \\
    \bottomrule
  \end{tabular}
  \caption{Allowed polarisation configurations between each hyperfine
  ground state Zeeman sub-levels.}
  \label{tab:raman_polarisation}
\end{table}
\begin{figure}[htpb]
  \centering
  \includegraphics[width=0.7\textwidth]{raman_spectrum}
  \caption[Raman transition spectrum]{Raman transition spectrum, obtained by scanning the beat
    frequency of the two Raman lasers. The transitions \(\ket{1,m_F}
  \rightarrow \ket{2,m_F'}\) are indicated at each observed peak.}
  \label{fig:raman_spectrum}
\end{figure}
\par\noindent
Each narrow transition from \(\ket{1,0}\) has broader
peaks on either side which are the Doppler-sensitive transitions
produced by counter-propagating beams.
The central peak is shown in more detail
in~\FigureRef{fig:raman_spectrum_inset}. The counter-propagating
transitions are shifted from the central co-propagating peak by \sivalue{-185}{\kilo\hertz} and
\(+\)\sivalue{215}{\kilo\hertz} respectively. This difference is
explained by the recoil shift $\Delta f_r = \frac{1}{2\pi}\frac{\hbar
k^2_\textnormal{eff}}{2m} = $ \sivalue{14.9}{\kHz}, which increases
the resonance frequency of both counter-propagating transitions.
After subtracting the recoil shift, the Doppler shifts of
$\pm$\sivalue{200}{\kHz} corresponding to
velocity of \sivalue{7.69}{\centi\metre\per\second}. The same Doppler shifts are also
observed in the peaks corresponding to the \(\ket{F=1,m_F = 0}
\rightarrow \ket{F=2,m_F=\pm 1}\) transitions. The counter-propagating transitions are
Doppler-broadened by the thermal velocity of the atoms along the direction
of the Raman beams. 
%The Doppler shift $\Delta f = \frac{2
%v}{\lambda}$, where $\lambda = $ \sivalue{780}{\nm} broadens the
%transition so that the FWHM is given by
%\begin{equation}
%  \Delta f_\text{FWHM} = 4 \sqrt{2 \ln{2}} \sqrt{\frac{k_B T}{m
%  \lambda^2}}
%\end{equation}
Fitting the transition to the lineshape expected
from a thermal distribution of atoms gives a temperature of
\sivalue{15}{\micro\kelvin} and \sivalue{13.5}{\micro\kelvin} from
each counter-propagating transition. At the time this spectrum was
measured, the molasses was not optimised to give the lowest
temperature.
\begin{figure}[htpb!]
  \centering
  \includegraphics[width=0.7\textwidth]{raman_spectrum_inset}
  \caption[\(\Delta m = 0\) transition spectrum.]{Transition spectrum showing the \(\Delta m = 0\) transition
  from \(\ket{1,0}\). The orange and green dashed lines are fits to a
Doppler-broadened lineshape for each of the counter-propagating
profiles.}
  \label{fig:raman_spectrum_inset}
\end{figure}

\subsection{Cancelling the Differential ac Stark
Shift}\label{subsec:light_shift}
It is worth considering the effects of ac Stark shifts on the atom
interferometer~\cite{Gauguet2008}. They are intrinsically related to the
effective Rabi frequency and as such, cannot be avoided. Each light
beam couples each hyperfine ground state to intermediate states in the
5$P_{1/2}$ and 5$P_{3/2}$ levels in Rubidium-87. The total ac Stark
shift of each level\footnote{The ac Stark shift term
in~\EquationRef{eq:raman_ac_stark} was defined assuming that each
field $\textbf{E}_j$ couples to only one ground state $\ket{j}$. Here,
we do not make that assumption.} is a sum over the shift from each
coupled excited state
\begin{equation}
  \Omega_j^\text{ac} = \sum_{ik} -\frac{|\Omega_{ijk}|^2}{2\Delta_{ijk}}
\end{equation}
in terms of the one-photon Rabi frequencies \(\Omega_{ijk}\) and detunings \(\Delta_{ijk}\) where the index $i = 1,2$ labels each light field, $j = 1,2$ labels
each hyperfine ground state and $k$ labels each excited state that is
coupled by a dipole transition to $j$. From the free evolution of the atom's internal state, there is a contribution to the phase along each path that is
proportional to the sum of the light shifts of each hyperfine ground
state $(\Omega_1^\text{ac} +
\Omega_2^\text{ac})$. For an interferometer pulse separation of $T=
$\sivalue{25}{\ms}, the maximal separation of each path is
$\frac{\hbar k_\text{eff}}{m} T$ = \sivalue{300}{\micro\m}. Close to
the centre of the Raman beams, there is a negligible variation of
intensity over this distance, so the sum of the ac Stark shifts of
each state should not lead to an observable interferometer phase
shift.
\par\noindent
The Raman resonance depends on the difference in energy of the two
states $\hbar(\omega_2 - \omega_1)$. The ac Stark shifts of each state
therefore leads to a contribution to the detuning that depends on the differential ac Stark shift \(\delta^\text{ac}
= \Omega_2^\text{ac} - \Omega_1^\text{ac}\). can lead to an observable
phase shift. Using the results from Ref.~\cite{Weiss1994} for \(\pi\)
and \(\frac{\pi}{2}\) pulses, the phase shift to a Mach-Zender type
interferometer is
\begin{equation}
  \Delta \Phi^\text{ac} =
  \frac{\delta_3^\text{ac}}{\Omega_\text{eff}} - \frac{\delta_1^\text{ac}}{\Omega_\text{eff}} 
 \label{eq:diff_phase}
\end{equation}
where \(\delta_3^\text{ac}\) and \(\delta_1^\text{ac}\) are the ac
Stark shifts of the last and first \(\frac{\pi}{2}\) pulses,
respectively, and $\Omega_\textnormal{eff}$ is the effective Rabi
frequency defined in~\EquationRef{eq:rabi_raman_sum}. Note that there
is no sum over the beam index $k$, because in making the rotating wave
approximation, terms which contain $\Delta_{21k}$ and $\Delta_{12k}$
are dropped as they oscillate much faster than the retained terms. 
\par\noindent
As the atoms fall
under gravity, they move through the Raman beam profile and experience
the interferometer pulses at different intensities. To the extent that
there is a light shift of the clock transition, this can produce a
non-zero interferometer phase. Fortunately, it is possible
to eliminate such a shift using an appropriate choice
of intensity and detuning of the Raman lasers. This can be seen by
first writing out the differential ac Stark shift
\begin{equation}
  \delta^\text{ac} = \Omega_1^\text{ac} - \Omega_2^\text{ac} = \sum_{ik}
  \frac{\lvert\Omega_{i1k}\rvert^2}{2\Delta_{i1k}} - \sum_{ik}
  \frac{\lvert\Omega_{i2k}\rvert^2}{2\Delta_{i2k}} 
  \label{eq:diff_shift}
\end{equation}
When both Raman beams are red-detuned from
all the one-photon transitions, both terms
in~\EquationRef{eq:diff_shift} are strictly negative. Therefore,
\(\delta^\text{ac}\) can be cancelled by choosing the correct
intensities for each Raman beam. A plot of \(\delta^{\text{ac}}\)
for various Raman beam intensities as a function of the ratio between
the two Raman beams is shown in~\FigureRef{fig:light_shift_ratio}. There
is a ratio at which the differential ac Stark shift cancels and is
independent of the total intensity. The one-photon detuning $\Delta_R$
is defined such that the laser frequency $\omega_2^l$ is detuned from
the \trans{2}{3} transition (in the absence of light shifts) by
$\Delta_R$. The ratio that cancels
\(\delta^\text{ac}\) for increasing \(\Delta_R\) is shown
in~\FigureRef{fig:light_shift_detuning}. When \(\Delta_R\) is
\sivalue{1.13}{\giga\hertz} below the \trans{2}{3} transition, this
ratio is maximised. The differential ac Stark shift is cancelled when
the intensity ratio of light driving \(\ket{1,0}\) transitions to
\(\ket{2,0}\) transitions is \(\mathcal{R} = 0.583\). 
\begin{figure}[htbp!]
	\centering
	\def\svgwidth{\columnwidth}
	\subfloat[][]{\scalebox{0.45}{\includegraphics{light_shift_ratio}}\label{fig:light_shift_ratio}}\\
\subfloat[][]{\scalebox{0.45}{\includegraphics{light_shift_detuning}}\label{fig:light_shift_detuning}}\\
	\caption[Differential ac Stark shift as a function of two-photon
  detuning and Raman beam intensities.]{The effects of the Raman beam
  intensities and detuning on the differential ac Stark shift
  \(\delta^\text{ac}\).
\textbf{(a)} shows \(\delta^\text{ac}\) as a function of the intensity ratio
\(\mathcal{R}\) between the light which drives transitions from
\(\ket{1,0}\) to the light that couples to \(\ket{2,0}\) for the
two-photon detuning of \(\Delta_R = -\sivalue{1.13}{\giga\hertz}\)
used in the experiment. Example
intensities for the \(\ket{2,0}\) light are
\sivalue{100}{\watt\per\metre\squared} (blue),
\sivalue{200}{\watt\per\metre\squared} (orange) and
\sivalue{300}{\watt\per\metre\squared} (green). \textbf{(b)} shows how
the ratio for which \(\delta^\text{ac} = 0\) varies as \(\Delta_R\)
increases. The dashed lines indicate the value of \(\Delta_R\) used in
the experiment and its corresponding ratio of 0.583.}
	\label{fig:light_shift_plots}
\end{figure}
\par\noindent
It is not straight-forward to directly measure the intensity of
each Raman beam on the atoms, so the transition spectrum was used to cancel the
differential ac Stark shift and determine when the intensities of the
lasers are set to the appropriate
ratio. Experimentally, this was done by adjusting the power of the
pump lasers for the master and slave SolsTiS lasers. When the master
is seeded with \sivalue{10}{\watt} and the slave with
\sivalue{6.5}{\watt}, the differential ac Stark shift is eliminated.
\FigureRef{fig:cancelled_light_shift} shows the transition spectrum using two different effective Rabi
frequencies, corresponding to \(\pi\) pulse times of
\sivalue{22.5}{\micro\second} and
\sivalue{45}{\micro\second}. In this instance, the frequency difference of the two
co-propagating peaks, shown in blue and orange, is less than \sivalue{1}{\kilo\hertz}.
\begin{figure}[htpb!]
  \centering
  \includegraphics[width=0.7\textwidth]{cancelled_light_shift}
  \caption[Raman transition spectrum after cancelling the differential
  ac Stark shift.]{Raman transition spectrum after cancelling the differential
    ac Stark shift. The blue curve shows a pulse with a \(\pi\) pulse
  time of \sivalue{22.5}{\micro\second} and the orange shows the
  spectrum using a less intense pulse with a $\pi$-pulse time of
\sivalue{45}{\micro\second}.}
  \label{fig:cancelled_light_shift}
\end{figure}
There is also a shift of \sivalue{1.4}{\mega\hertz} from
\(f_\text{hfs}\).
This is a result of a
second-order Zeeman shift and corresponds to a field strength of
\sivalue{1.56}{\gauss}.
\subsection{Velocity-Selective Pulse}\label{subsec:vel_select}
The accelerometer has to use the velocity-sensitive transitions
induced by counter-propagating beams in order to be sensitive to
acceleration. In order for all the atoms to experience the same
desired pulse areas of $\pi/2-\pi-\pi/2$, it is necessary for the
Doppler width to be much less than the natural width arising from the
duration of the Raman pulses. In this apparatus, the laser cooling
gives a temperature of \sivalue{6}{\micro\kelvin}, for which the Doppler
width (FWHM) is \(\sigma_f = \frac{4 \sqrt{2 \ln(2)}}{\lambda}\sqrt{\frac{k_b T}{m}}
\approx\) \sivalue{140}{\kilo\hertz}. 
A $\pi$-pulse duration of \sivalue{6}{\micro\second} gives a
linewidth close
to this Doppler width, but the intensities required for this are above
what is attainable with our Raman laser. 
\par\noindent
We therefore reduce the Doppler width of the participating atoms
by first applying a long Raman pulse to select a subset of the population
with a narrower velocity spread~\cite{Moler1992}. This
velocity-selective pulse gives a narrower Doppler linewidth than the
subsequent shorter
interferometer pulses thereby ensuring that the Rabi frequencies are
reasonably homogeneous across the cloud of selected atoms.
\par\noindent 
Starting with a velocity distribution of atoms
described by a 1-D Maxwell-Boltzmann distribution all occupying the
\(\ket{1,0}\) state, the
population in \(\ket{2,0}\) after applying a Raman pulse is
distributed according to
\begin{equation}
  \text{P}_{\ket{2,0}}(v) = \frac{\Omega_\textnormal{eff}^2}{\Omega_\textnormal{eff}^2 + \delta^2}
  \sin\left(\sqrt{\Omega_\textnormal{eff}^2+\delta^2}\;\tau\right)^2 p(v)
  \label{eq:vel_selected_dist}
\end{equation}
where \(\delta\) is the Raman detuning defined
in~\EquationRef{eq:raman_detuning}, \(p(v) = \sqrt{\frac{m}{2\pi
k_B T}} e^{-\frac{m v^2}{2 k_B T}}\) is the velocity distribution and
\(\Omega_\textnormal{eff}\) is the effective Rabi frequency defined
in~\EquationRef{eq:rabi_raman_sum}. \FigureRef{fig:vel_selected_dist}
shows the expected
distribution of Doppler shifts in the selected atoms after a
\sivalue{6}{\micro\K} cloud has been driven by a $\pi$ pulse with a duration of
\sivalue{40}{\micro\second}.
The population that is stimulated
has a mean velocity shifted by twice the recoil velocity. In this
instance, the rms
Doppler shift is \(\sigma_f
=\)\sivalue{19.7}{\kilo\hertz}.
\begin{figure}[htpb!]
  \centering
  \includegraphics[width=0.7\textwidth]{vel_selected_dist.pdf}
  \caption[Simulated velocity distribution after a Raman $\pi$
  pulse.]{Calculated distribution of frequency shifts in the ensemble
    selected from a \sivalue{6}{\micro\kelvin}
  cloud by a \sivalue{40}{\micro\second} Raman \(\pi\)
pulse. Blue curve: atoms not selected. Orange curve: atoms selected.
Note the recoil shift of the selected atoms in addition to their
Doppler shifts. }
  \label{fig:vel_selected_dist}
\end{figure}
\subsubsection{Velocity-Selected Distribution}
The velocity distribution of atoms after the velocity selective pulse
can be measured using a second Raman pulse as a probe. This probe must
be much longer duration than
the velocity-selection pulse so that it gives even narrower
resonances. 
\FigureRef{fig:vel_select_chirp} shows a measurement of the transition
spectrum close to the peak of the Doppler-sensitive transition that is
greater in frequency than the Doppler-insensitive peak. The orange
curve shows the spectrum obtained before velocity selection, using a single Raman $pi$ pulse of
duration \sivalue{40}{\micro\s}. The blue curve, which illustrates the
distribution of atoms after velocity selection, is obtained by first
applying a
\sivalue{40}{\micro\second} \(\pi\) pulse with a Raman beat frequency
\(f_v = \) \sivalue{6834.51}{\mega\hertz}. This prepares atoms in
\(\ket{1,0}\), then the atoms which remain in
\(\ket{F=2}\) are blown away. After \sivalue{10}{\milli\second}, a
\sivalue{80}{\micro\second} \(\pi\) pulse transfers some of the
remaining population back into \(\ket{2,0}\). The frequency of the
probe pulse is varied by chirping the Raman laser beat frequency. In
this instance, the power of the \sivalue{80}{\micro\s} pulse was not tuned to give a \(\pi\)
pulse area so the measured population is not indicative of the maximum
driven by the Raman transition. The red curve shows a fit to a
Doppler-broadened lineshape, which gives an effective temperature
of around \sivalue{1}{\micro\K}. Comparing this to the peak from
before velocity selection, it is clear that the velocity
distribution of the selected atoms is much narrower. 
\begin{figure}[htpb!]
  \centering
  \includegraphics[width=0.7\textwidth]{RamanChirp.pdf}
  \caption[$\ket{F=2}$ population after a velocity-selective Raman
  $\pi$ pulse.]{\(\ket{F=2}\) population after a Raman pulse at a frequency
    \(f_v =\)\sivalue{6834.51}{\mega\hertz} transfers atoms to
    \(\ket{1,0}\). This distribution is probed by
  applying a narrow pulse at a frequency \(f_{R}\). The population
measured in \(\ket{F=2}\) is shown in blue. The red dashed line is a
fit to the Doppler-broadened
transition peak. The orange curve shows
transition spectrum (without velocity selection) from a $pi$-pulse
of duration \(\tau = \)\sivalue{40}{\micro\s}.}
  \label{fig:vel_select_chirp}
\end{figure}
\subsection{Interferometer Pulses}\label{subsec:int_pulses}
The power settings needed to make Raman $\pi$ and $\pi/2$ pulses were
empirically determined by observing Rabi oscillations. These are shown
in~\FigureRef{fig:rabi_oscillation}. For a $\pi$ pulse, the Raman
laser power was
set so that the Rabi oscillation reached its maximum after a pulse duration of
\(\tau = \)\sivalue{15}{\micro\s}. It is clear that the oscillations
are rapidly damped. This dephasing rate depends on the
time at which the pulse is applied. Since the atoms are at different
positions in the beam, this suggests that the dephasing is caused by a
spatial variation of the Rabi frequency. This is largely a result of
irregularities in the Raman beam intensity profile from defects in
the two small aspheric lenses. The Gaussian intensity distribution of each beam
is unlikely to cause such a fast dephasing, since the atoms remain
close to the centre of the beam. At longer pulse durations,
spontaneous decay from the intermediate states of the Raman transition
becomes apparent.
\begin{figure}[htpb]
  \centering
  \includegraphics[width=0.8\linewidth]{SinglePulses.pdf}
  \caption[Rabi oscillations between the $\ket{1,0}$ and $\ket{2,0}$
    states.]{Rabi oscillations between the $\ket{1,0}$ and $\ket{2,0}$
    states. Each pulse occurs at different times after the \ac{mot} is
    released. The times shown are \sivalue{13}{\ms} (blue),
  \sivalue{23}{\ms} (orange) and \sivalue{33}{\ms} (green). Full
  population transfer is not observed due to the atoms in $\ket{1,\pm
1}$ which have not been removed. }
\label{fig:rabi_oscillation}
\end{figure}
\section{Noise}\label{sec:atomint_sensitivity}
This section presents a discussion of the identified sources of
noise in the interferometer and their effects on the sensitivity to
accelerations. It begins with a discussion of the noise that arises
from detecting atoms in~\SectionRef{subsec:detection_noise}. This
defines an uncertainty in measuring the occupation probability, which then used to
express the corresponding acceleration uncertainty
in~\SectionRef{subsec:accel_uncert}. The Allan variance used to characterise
the stability of the interferometer signal is defined
in~\SectionRef{subsec:allan_variance}. The sensitivity function and an analytic method
for calculating the influence of random phase fluctuations is given
in~\SectionRef{subsec:sens_func}. This is applied to phase noise from
the Raman laser in \SectionRef{subsec:laser_phase} and external
vibrations in~\SectionRef{subsec:vibration_noise}. Finally, the
section concludes with a summary of the identified sources of noise
in~\SectionRef{subsec:noise_sources}.
\subsection{Detection Noise}\label{subsec:detection_noise}

Each measurement of the number of atoms has an uncertainty due to
random processes that influence the voltage measured by the detector. These errors
combine to give an uncertainty in the interferometer phase and
hence, acceleration. It is worth distinguishing between the different
sources of noise in measuring the occupation probability
$\text{P}_{\ket{F=2}}$. Fluctuations in the number of atoms and detected
photons lead to an uncertainty in the measured voltages that
correspond to $N_2$ and $N_\text{Tot}$ 
\nocite{Rocco2014}. However, the main fundamental limitation
is the noise that arises when an atom is projected into either the
$\ket{1,0}$ state or the $\ket{2,0}$ state. In what follows, these
three sources of detection noise are discussed in the context of the
detection sequence previously described
in~\SectionRef{subsec:detection_sequence}. Naturally, we aim to have a
good enough photo-detector that this is not a limiting source of
noise.
\subsubsection{Atom and Photon Number Fluctuations}
The random processes by which atoms are loaded and cooled in the
initial stages of the experiment, as well as the random direction a
photon is spontaneously emitted mean that both the total number of
atoms after interferometry and the number of photons arriving at the
detector within a given time duration will have shot noise
fluctuations.  
Thus, the mean voltage measured by the detector over an integration
time of $\tau$ should have a shot-noise variation of
\begin{equation}
  \sigma_\text{v}^2 = \alpha^2 \eta^2 R_\text{sc}^2 n_\text{at}\left(1
  + \frac{1}{q \eta R_\text{sc} \tau} \right)
  \label{eq:atom_photon_noise}
\end{equation}
where $\alpha = \hbar \omega G$ converts the photon rate incident on
the photodiode to an output voltage and $q$ is the quantum efficiency
of the photodiode. The first and second terms represent the number
fluctuations of atoms and photons respectively. The denominator $q
\eta R_\text{sc} \tau$ is just the number of photons detected per
atoms. 
\par\noindent
As previously discussed
in~\SectionRef{subsec:photodiode_setup}, around 7 photons per atom
so the photon shot noise is much less than that of the atoms. For few
million atoms that we use, the predicted statistical noise is
approximately \sivalue{0.1}{\milli\volt}. In practice however, the
measured shot-to-shot noise is much higher than that --
typically \sivalue{3}{\milli\volt}. This is due to fluctuations in the
number of atoms collected in the \ac{mot}, which arise in turn from
fluctuations of the \ac{mot} laser intensity and polarisation.
Fortunately, the method we use for determining acceleration makes us
insensitive to the atom number fluctuations. The occupation probability $\text{P}_{\ket{F=2}}$ is determined by
taking the ratio of
$N_2$ and $N_\text{Tot}$. Because these are both measured on the same
group of atoms, the number of atoms cancels out in the ratio and the
shot-to-shot noise is suppressed in
$\text{P}_{\ket{F=2}}$. 
\subsubsection{Quantum Projection Noise}
Although the determination for
$\text{P}_{\ket{F=2}}$ does not suffer from atom number fluctuations,
there is an uncertainty arising from the
projection of an atom's internal state onto 
the $\ket{2,0}$ state when we detect $N_2$. 
Consider a number of atoms $n_{10}$ prepared in state $\ket{1,0}$. Let
the probability of their making a transition to $\ket{2,0}$ be
$\langle p_2\rangle$. Then the number arriving in state $\ket{2,0}$ is
$n_{10} (\langle p_2 \rangle \pm \langle p_2 \rangle \left( 1- \langle
p_2 \rangle \right)$, where the uncertainty is the quantum projection
noise~\cite{Bollinger1996}. Therefore, the measured transition probability is given by
\begin{align}
  \text{P}_{\ket{F=2}} &= \frac{\langle p_2 \rangle n_{10} \pm \sqrt{n_{10}
  \langle p_2 \rangle \left( 1- \langle
p_2 \rangle \right)}}{n_{10}(1+\epsilon)}
  \nonumber \\
                &= \frac{\langle p_2 \rangle}{1+\epsilon} \pm
                \frac{1}{1+\epsilon} \sqrt{\frac{\langle p_2 \rangle
                    \left(1- \langle
                p_2 \rangle \right)}{n_{10}}} 
                \label{eq:prob_noise}
\end{align}
where as discussed in~\SectionRef{subsec:atom_number_bias}, $\epsilon$
accounts for the unwanted atoms left in $m_F = \pm 1$ 
(and it is assumed for simplicity that this ratio
does not fluctuate).  
For a given interferometer phase $\Phi$,
$\langle p_2 \rangle = \sin^2(\Phi/2)$, giving an interferometer
fringe pattern, including the quantum projection noise, of
\begin{equation}
  \text{P}_{\ket{F=2}} = \frac{1-\cos(\Phi)}{2(1+\epsilon)} \pm
  \frac{1}{2(1+\epsilon)}\frac{\sin(\Phi)}{\sqrt{n_{10}}}
\end{equation}
A useful measure of the interferometer's sensitivity is given by the
ratio of the noise to the fringe amplitude which is
\begin{equation}
  S = \frac{\sin(\Phi)}{\sqrt{n_{10}}}
  \label{eq:sens_proj}
\end{equation}
and is independent of $\epsilon$. The interferometer is most sensitive
to acceleration at the mid-point
of a fringe where $\Phi = \frac{\pi}{2}$. There, the quantum
projection noise has its maximum value relative to the fringe
amplitude of $1/\sqrt{n_{10}}$. The corresponding uncertainty in
acceleration is derived in~\SectionRef{subsec:accel_uncert}. Note that
the photon shot noise, being small compared with the noise from atom
number fluctuations, is also small compared with projection noise.

\subsubsection{Photodiode Technical Noise}
%In order for the detection noise
%be dominated by the quantum projection noise, the detector must be sensitive
%enough that it has a \ac{nep} much lower than the noise in the optical
%power detected. The \ac{nep} of a detector is defined as the
%equivalent optical power which gives a signal-to-noise ratio of 1
%after an integration time of \sivalue{0.5}{\second}. The photodiode
%signal is amplified so it is convenient
%to express the technical noise as a voltage density, by multiplying it by the
%amplifier gain. Hence, the
%voltage density of a detector whose sensitivity is equivalent to the
%voltage corresponding to the quantum projection noise for an
%integration time \(\tau_D\) is given by
%\begin{equation}
%  V^2_\text{at} =
%  \frac{1}{2\tau_D}\eta \hbar \omega G R_\text{sc}\Delta N_\text{proj}
%  \label{eq:nep}
%\end{equation}
%where $\Delta N_\text{proj} = \sqrt{N_\text{proj}}$ is the variance in
%the number of atoms detected due to the projection of atoms onto the
%states $\ket{1,0}$ and $\ket{2,0}$. If the signal is detected for $\tau_D =$ \sivalue{200}{\micro\s},
%multiplicative factor that relates $V_\text{at}$ to
%$\Delta N_\text{proj}$ is \sivalue{6.7e-12}{\volt\squared\per\hertz}.
%\par\noindent
The quantum projection noise is an irreducible noise level coming from
the atoms\footnote{This can be reduced by squeezing the pseudospin
  representing the clock transition coherence~\cite{Bollinger1996},
but we do not do that here.}. Not wishing to have more noise than
that, we should choose a suitably quiet detector. 
Technical noise in the detector typically arises from multiple electronic processes -- such as Johnson noise and shot noise in
the current~\cite{Howard2002}. The technical noise of the detector was determined by measuring the output voltage when no light is
collected. A plot of the power spectral density of the photodiode is
shown in~\FigureRef{fig:noise_source_psd} taken
with a sampling frequency of \sivalue{200}{\kilo\hertz}. The
photodiode was covered and the output voltage was sampled for
\sivalue{2}{\second}. The power spectral density has been calculated
using Welch's method~\cite{Welch1967}. The data are partitioned before
calculating the Fourier transform of each subset and taking the
average. This has the effect of reducing the variance in the estimated
power spectrum at the expense of reducing the frequency resolution.
Below \sivalue{10}{\kilo\hertz} the power spectral density is close
to uniform with a value of around
\sivalue{5e-13}{\volt\squared\per\hertz}, which corresponds to a
noise-equivalent power of \sivalue{391}{\femto\W\per\hertz\tothe{1/2}}.
For higher frequencies, the power spectral density starts to
increase. 
\par\noindent
The variance of the detector noise is given by the integral of the
noise power spectral density over the bandwidth $B$, which is related
to the integration time $\tau$ according to $B = 1/(2
\tau)$\sivalue{}{\hertz}. The blue solid line
in~\FigureRef{fig:noise_source_int} shows how the detector noise
variance decreases as the integration time is increased. In the same
plot, the dashed red line shows the variance of the signal voltage
resulting from quantum projection noise when $n_{10} =$ \num{3e6}, as
is typical in the apparatus. 
The
dot-dashed line corresponds to a detection time of
$\sivalue{200}{\micro\s}$ which is normally the time we use. One can
see that the detector noise is a little larger than the quantum
projection noise. Adding the two variances and taking the square root,
we find that the quantum projection noise of \sivalue{40}{\micro\volt}
is inflated by the detector noise to \sivalue{90}{\micro\volt}. In
future, it would be good to replace the detector by one with lower
noise. It might also be possible to increase $\tau$, but in practice
this causes an undesirable deterioration of the signal.
\begin{figure}[htpb!]
  \centering
  \includegraphics[width=0.7\textwidth]{noise_source_psd}
  \caption[Power spectral density of the amplified photodiode voltage.]{Power spectral density
    of the voltage from the amplified photodiode signal
  sampled for \sivalue{2}{\second} at a rate of
\sivalue{200}{\kilo\hertz}.  
}
  \label{fig:noise_source_psd}
\end{figure}
\begin{figure}[htpb!]
  \centering
  \includegraphics[width=0.7\textwidth]{noise_source_int}
  \caption[Detector noise vs. integration time.]{Detector noise
    variance as a function of integration time, shown in blue. The red
    curve is the variance in the voltage expected from the quantum
    projection noise of \num{3e6} atoms. The dot-dashed line indicates
    the integration time of \sivalue{200}{\micro\s} that is presently
  used.}
  \label{fig:noise_source_int}
\end{figure}


%\par\noindent
%Averaging the detection signal over a time \(\tau\) has the effect of
%filtering the signal above the Nyquist frequency \(f_n = 1/(2 \tau)\).
%The variance in the averaged voltage over successive shots
%i.e. the Allan variance, is related to the power spectral density as
%follows
%\begin{equation}
%  \sigma^2_\text{av}(\tau) = 2 \int_0^\infty \frac{\sin(\pi \tau f)^4}{(\pi \tau f)^2}S(f)\;\mathrm{d}\;f
%  \label{eq:av_psd}
%\end{equation}
%Using~\EquationRef{eq:av_psd}, it is possible to determine the
%detection time required to reduce the shot-to-shot variance below the
%quantum projection noise level. \FigureRef{fig:detecion_av} shows the Allan deviation for increasing
%integration time. Above it shows the $\tau^{-1/2}$ dependence
%characteristic of uncorrelated white noise. The red and greed curves
%show the voltages corresponding to the quantum projection noise
%for the same numbers of atoms used
%in~\FigureRef{fig:noise_source_psd}. At the integration time of
%\sivalue{200}{\micro\second}, the detector noise
%is close to \sivalue{5}{\micro\volt} -- well below the quantum
%projection noise level.
%\begin{figure}[htpb!]
%  \centering
%  \includegraphics[width=0.7\textwidth]{detection_av}
%  \caption[Allan deviation of Photodiode voltage.]{Allan deviation of
%    the amplified photodiode voltages for increasing
%  integration time \(\tau\). The red and green dashed lines are the
%quantum projection noise levels for the atom numbers used
%in~\FigureRef{fig:noise_source_psd}. The black dot-dashed line indicates the
%integration time of \sivalue{200}{\micro\second} used in the
%experiment.}
%  \label{fig:detecion_av}
%\end{figure}

\subsubsection{Summary of Detection Noise}
It is worth summarising the various contributions to the noise in the
occupation probability $\text{P}_{\ket{F=2}}$. Fluctuations in the
atom number and detected photon number lead to shot noise
contributions to the voltages corresponding to $N_2$ and
$N_\text{Tot}$. However, we detect many photons per atom and in taking
the ratio of the two voltages, the contribution of variations in the
total atom number to $\text{P}_{\ket{F=2}}$ is largely suppressed.
Thus the main fundamental noise comes from the quantum projection
noise. There is also technical noise
in the detector. 
At present, the technical detector noise is slightly greater than the projection
noise. In the future, it would be good to replace the detector with a
quieter one.

\subsection{Acceleration Uncertainty}\label{subsec:accel_uncert}
%The noise in measuring the occupation of each state determines the
%accuracy to which the interferometer phase, and hence acceleration,
%can be measured.
%For simplicity, it is assumed that the fringe contrast $C$ is
%independent from the mean probability of detecting atoms in
%$\ket{F=2}$, $\text{P}_0$. This is valid if the contrast is reduced due to
%inhomogeneities in the Rabi frequencies of each pulse and if the mean
%probability is reduced from 0.5 by the presence of background $m_F=\pm
%1$ atoms,
%which can never occupy $\ket{F=2}$. Under these
%assumptions~\EquationRef{eq:prob_measurement} yields the following variance
%\begin{equation}
%  \sigma_\text{P}^2 = \sigma_{P_0}^2 + \frac{1}{4} \sigma_C^2
%  \cos(\Phi)^2 + \frac{C^2}{4}\sigma_{\Phi}^2 \sin(\Phi)^2
%\end{equation}
%At the mid-point of the fringe, $\Phi = \pi/2$, this variance
%becomes
%\begin{equation}
%  \sigma_\text{P}^2 = \sigma_{P_0}^2 + 
%  \frac{C}{4}\sigma_{\Phi}^2 
%\end{equation}
%so if $\text{P}_0$ and $C$ do not vary, the ratio of $\sigma_P$ to the
%fringe amplitude $C/2$ is
%given by
%\begin{equation}
%  S = \sigma_{\Phi}
%\end{equation}
%Returning to the interferometer sensitivity defined
%in~\EquationRef{eq:sens_proj}, we can identify the maximum sensitivity
%to accelerations as 
%\begin{equation}
%  \sigma_a = \frac{1}{\sqrt{n_{10}} k_\text{eff} T^2}
%\end{equation}
The measurement of $\text{P}_{\ket{F=2}}$ is interpreted as a
measurement of acceleration assuming that $\text{P}_{\ket{F=2}}$ has
the form $\text{P}_0 - (C/2)\cos(\Phi)$. In the sensitive region of
the fringe pattern, where the slope is highest, we may rewrite $\Phi =
(2n - 3/2)\pi + \theta$. Here $n$ is the order number of the fringe.
That gives $\text{P}_{\ket{F=2}} = \text{P}_0 + (C/2) \sin(\theta)$,
or $\text{P}_0 + (C/2)\theta$ for small $\theta$. Similarly, on the sides of the fringe
with the most negative slope, we have $\Phi = (2n-1/2)\pi + \theta$ which
yields $\text{P}_0 - (C/2)\theta$ for small $\theta$. The
accelerometer is calibrated by measuring $\text{P}_0$ and $C$. Then,
recalling from~\SectionRef{subsec:theory_path} that $\Phi =
k_\textnormal{eff}
a T^2$, it is straightforward to determine $a$ as long as the
order number is known. Operating near these points of large slope, the
noise $\sigma_{P_2}$ translates to acceleration noise $\sigma_a =
\frac{2}{C k_\textnormal{eff} T^2} \frac{1}{(1+\epsilon)\sqrt{n_{10}}}$. 
So it is clear that the uncertainty in measuring the acceleration is
made small by
increasing the number of atoms or by increasing the time between
interferometer pulses. The wavelength of the Rubidium transition fixes
the value of $k_\text{eff}$. Taking the best possible case in which
$C=1$, $\epsilon = 0$ and there is no detector noise, our current
number of \num{3e6} atoms and our current interaction time of
\sivalue{25}{\milli\second} together give $\sigma_a =
$ \sivalue{100}{\nano\meter\per\second\squared}. In terms of the
interferometer phase, that corresponds to a noise $\sigma_\Phi =
\frac{2}{\sqrt{n_{10}}} = $ \sivalue{1}{\milli\radian}. At present
however, the phase lock for the Raman beams has approximately
\sivalue{10}{\milli\radian} of noise, leading us to expect an
uncertainty to acceleration of order
\sivalue{1}{\micro\meter\per\second\squared} in practice, as discussed
further in~\SectionRef{subsec:sens_func} below.
\subsection{The Allan Variance}\label{subsec:allan_variance}
So far we have only considered random noise, for which the variance
converges over a large enough sample. For correlated noise, this is
not necessarily the case and then it is useful to measure the Allan
variance.This measures the mean square difference between successive groups of
$N$ measurements, each member of the group being averaged over a time
$\tau$. Thus each average is over a time $t = N \tau$ and the Allan
variance is~\cite{Allan1966}. 
\begin{equation}
  \sigma_\text{av}^2(t) = \frac{1}{2}
  \left\langle
  \left(\frac{1}{N}\sum_{k=0}^{N-1}a_{n+1}-\frac{1}{N}\sum_{k=N}^{2N-1}a_n\right)^2\right\rangle
  \label{eq:allan_var_mean}
\end{equation}
%For instance, in the case of white
%noise, the uncertainty in each measurement is uncorrelated
%and the Allan variance becomes
%\begin{equation}
%  \sigma^2_\text{av}(t) = \frac{1}{N}\sigma^2_\text{av}(\tau)
%\label{eq:allan_var_simp}
%\end{equation}
%This is typically the case over short timescales. More generally, the
The Allan variance can be used to identify the timescales that
characterise the behaviour of the bias in an
accelerometer~\cite{El-Sheimy2008}. At short
timescales, the Allan variance is dominated by white noise if the
variance in successive measurements is uncorrelated. 
At longer timescales the bias instability -- fluctuations of
the bias value -- causes the Allan variance to reach a minimum value.
Finally, at still longer timescales, the bias drift -- a long-term
change in the bias value -- causes
an increase in the Allan variance. \TableRef{tab:avar} summarises
these three processes and their characteristic dependences on the
averaging time $t$.
\begin{table}[htpb]
  \centering
  \begin{tabular}{cc}
  \toprule
    Process & $t$ dependence \\ 
    \midrule
    White noise & $t^{-1}$ \\
    Bias instability & $t^{0}$ \\
    Bias drift & $t^{2}$\\
    \bottomrule
  \end{tabular}
  \caption{Allan variance characteristic timescales.}
  \label{tab:avar}
\end{table}

\subsection{The Sensitivity Function}\label{subsec:sens_func}
The sources of noise presented thus far arise from various 
uncertainties in measuring the number of atoms from their
fluorescence. When using the atom interferometer to measure
accelerations, there is additional noise from fluctuations of the
Raman phase and from real vibration noise. I consider first the effect
on the interferometer phase $\Phi$ from fluctuations of the Raman
phase $\phi$.
\par\noindent
Let a small fluctuation $\delta \phi$ at time $t$ produce a
fluctuation $\delta \Phi$ of the interferometer phase. Making a
first-order Taylor expansion, we can write
\begin{equation}
  \delta \Phi = \lim_{\delta\phi \rightarrow 0} \left(\frac{\delta
  \Phi}{\delta \phi(t)}\right) \delta \phi(t) = g(t) \delta \phi(t)
\end{equation}
where the last step defines the sensitivity function
$g(t)$~\cite{Dick1987}. If $\phi$ fluctuates continuously, then the
total shift of $\Phi$ is
\begin{align}
  \Delta\Phi &= \int_{-\infty}^\infty g(t)\;\mathrm{d}\phi(t)
  \nonumber\\
  &= \int_{-\infty}^\infty g(t)\;\frac{\mathrm{d}\phi}{\mathrm{d} t}\;
  \mathrm{d}t
  \label{eq:phase_contrib}
\end{align}
Consider our \(\pi/2-\pi-\pi/2\) interferometer with pulses 
of durations \(\tau, 2\tau, \text{and} \tau\). Defining \(t=0\) to be at
the middle of the sequence, the first pulse begins at time $-(2 \tau +
T)$ and the last pulse ends at $(2\tau + T)$. Outside that region the
Raman phase has no influence on the interferometer phase and $g(t) =
0$. Within that region, we have for positive times~\cite{Cheinet2008}.
\begin{equation}
    g(t) = 
    \begin{cases}
      \sin ( \Omega t) & 0<t<\tau  \\
      1 & \tau_R <t<\tau +T \\
      \sin (\Omega  (t-T)) & \tau +T<t<2 \tau +T
  \end{cases}
\label{eq:sensitivity_interferometer}
\end{equation}
and for negative times we have $g(t) = - g(-t)$.
\subsection{The Interferometer Transfer Function}\label{subsec:laser_phase}
The sensitivity function can now be used to derive the transfer
function, which defines the response of the interferometer phase to
the Raman phase. We first consider the response to a phase modulation at a
fixed frequency $\omega$ and an arbitrary phase offset $\psi$, which
is written as
\(\phi(t) = A\cos(\omega t + \psi)\). Substituting this into
\EquationRef{eq:phase_contrib} we obtain
\begin{equation}
  \Delta\Phi = - A \omega \cos(\psi) \int_{-\infty}^\infty g(t) \sin(\omega
  t) \mathrm{d} t
  \label{eq:phase_fourier}
\end{equation}
where the term proportional to \(\cos(\omega t)\) has been dropped,
since \(g(t)\) is an odd function. The integral
in~\EquationRef{eq:phase_fourier} is the Fourier
transform of \(g(t)\). Let us define
\begin{equation}
  H(\omega) = -\omega \int_{-\infty}^{\infty} g(t) \sin(\omega t),
  \mathrm{d} t,
  \label{eq:sensitvity_fourier}
\end{equation}
then we can write~\EquationRef{eq:phase_fourier} as
\begin{align}
  \Delta\Phi = A \cos(\psi) H(\omega)
  \label{eq:interfometer_fourier}
\end{align}
Using~\EquationRef{eq:sensitivity_interferometer}
we find that~\nocite{Canuel2007}
\begin{equation}
  H(\omega) = \frac{4 \omega
  \Omega}{\omega^2-\Omega^2}\sin\left(\frac{\omega(T+2\tau)}{2}\right)\left(\cos\left(\frac{\omega(T+2\tau)}{2}\right)
  + \frac{\omega T}{2}\sin \left(\frac{\omega T}{2}\right)\right).
  \label{eq:sens_fourier_full}
\end{equation}
The transfer function $|H(\omega)|^2$ determines the
spectral density $\Delta\Phi(\omega)$ of the interferometer phase
noise from Raman phase noise at angular frequency $\omega$.
\par\noindent
\FigureRef{fig:transfer_function} plots $|H(\omega)|^2$ for
\(T= \) \sivalue{20}{\milli\second} and \(\tau
= \) \sivalue{20}{\micro\second}. The asymptotic properties of this
function can be summarised as follows:
\begin{itemize}
  \item At low frequencies \(\omega \ll \Omega\), the transfer
    function is well approximated by 
    \begin{equation}
      |H(\omega)|^2 \approx 16 \sin^4 \left(\frac{\omega T}{2}\right)
    \end{equation}
    which is a periodic function that is zero at frequency multiples
    of \(1/\pi T\) \\
  \item At frequencies \(\omega \gg \Omega\), the transfer function is
    \begin{equation}
      |H(\omega)|^2 \approx 4 \frac{\Omega^2}{\omega^2}\sin^2
      \left(\omega T\right)
    \end{equation}
    which is zero when $\omega$ is an integer multiple of $\pi/T$.
\end{itemize}
\begin{figure}[htpb!]
  \centering
  \includegraphics[width=0.7\textwidth]{transfer_func_laser.pdf}
  \caption[Transfer function for Raman phase noise.]{Transfer function
  for Raman phase noise. Here, the separation between pulses is \(T =
  \)\sivalue{20}{\milli\second} and the \(\pi/2\) pulse time is \(\tau
=\)\sivalue{20}{\micro\second}.}
  \label{fig:transfer_function}
\end{figure}
\par\noindent
Returning to an arbitrary Raman phase with a power spectral density
$S_\phi(\omega)$, the power spectral density of the interferometer
phase is given by $S_\phi(\omega) |H(\omega)|^2$. Therefore, its
variance is given by the following integral
\begin{equation}
  \sigma_{\Delta\Phi}^2 = \frac{1}{2}\int_{-\infty}^\infty
    S_\phi(\omega)|H(\omega)|^2 \mathrm{d}\omega 
  \end{equation}
where the factor of $1/2$ is obtained by averaging over
$\cos^2(\psi)$.
\FigureRef{fig:laser_phase_drift} shows the drift in the Raman phase
measured over a period of \sivalue{20}{\s}, which is obtained by
demodulating the beat-note between the two Raman frequencies. There is a noticeable
variation over long timescale, on the order of \sivalue{10}{\radian}.
Fortunately, this is not of great concern to the interferometer, which is only sensitive to changes in the Raman phase
over the duration between pulses. \FigureRef{fig:laser_phase_drift_short} shows the
variation of the Raman phase over the first \sivalue{50}{\ms}, which
is the total interrogation time for a pulse separation of $T = $
\sivalue{25}{\ms}, assuming that the duration of each pulse is
negligible. At this timescale, the Raman phase
is much more stable. The power spectral density of the interferometer
phase resulting from Raman phase noise is shown
in~\FigureRef{fig:laser_int_psd}. It is clear that the interferometer
phase is most sensitive to variations in the Raman phase at the
frequency corresponding to the inverse of the interrogation time
$1/(2T) = $\sivalue{20}{\Hz}. The bandwidth of the Raman laser
\ac{pll} is much greater than the 
bandwidth of \sivalue{5}{\kHz} that corresponds to the
\sivalue{10}{\kHz} sampling frequency as evidenced by the fact that
the Raman laser \ac{pll} ensures that
the power spectral density of the interferometer phase is suppressed.
A measurement of the Raman phase power spectral density at higher frequencies has
not yet been undertaken. It is expected that the phase noise spectral
density will increase as the frequency approaches the bandwidth of
the~\ac{pll}. Since this is much greater than both the Rabi frequency
and the inverse of the interrogation time, it is unlikely that the
high-frequency Raman phase noise will contribute to the interferometer
phase noise.
\begin{figure}[htpb!]
  \centering
\subfloat[][]{\includegraphics[width=0.5\textwidth]{laser_phase_drift}\label{fig:laser_phase_drift}}
\subfloat[][]{\includegraphics[width=0.5\textwidth]{laser_phase_drift_short}\label{fig:laser_phase_drift_short}}
\\
\subfloat[][]{\includegraphics[width=0.5\textwidth]{laser_int_psd}\label{fig:laser_int_psd}}
\caption[Interferometer phase noise due to the Raman phase]{\textbf{a}
  shows the variation in the Raman phase over a period of
  \sivalue{20}{\s} sampled at a frequency of \sivalue{10}{\kHz}.
  \textbf{b} shows the first \sivalue{50}{\ms} of this same variation.
  \textbf{c} is the resultant power spectral density of the
interferometer phase.}
  \label{fig:laser_phase_noise}
\end{figure}
\par\noindent
The interferometer is sensitive to the average Raman phase between
each pulse. Over this time, the Raman phase noise may be correlated,
so it is natural to characterise the shot-to-shot variation of the
interferometer phase using the
Allan variance~\cite{Gouet2008}. The Raman phase
variation is first partitioned into sets of duration \sivalue{2}{\s}
to reduce the standard error on the estimated Allan variance. For a
given pulse separation $T$, each set is subdivided into subsets of
duration $2T$. 
From each subset, an interferometer phase shift is calculated using~\EquationRef{eg:phase_contrib}.
\FigureRef{fig:allan_dev_m2} shows the Allan deviation of the
interferometer phase as a function of the pulse separation time. At an interferometer pulse
separation of
\(T = \) \sivalue{25}{\milli\second}, the phase noise in the Raman
laser results in a shot-to-shot variation in the interferometer phase
of close to \sivalue{10}{\milli\radian}. This is an order of magnitude
larger than the previously discussed quantum projection noise and photodiode technical
noise and therefore limits the acceleration sensitivity in the absence
of external effects such as vibrations. This ultimately depends on the
stability of the reference oscillator in the Raman laser \ac{pll}.
Replacing this with one of lower phase noise will act to reduce the
shot-to-shot variation of the interferometer phase. 
\begin{figure}[htpb!]
  \centering
  \includegraphics[width=0.7\textwidth]{allan_variance_laser.pdf}
  \caption[Allan deviation of the Raman laser phase difference.]{Mean Allan
    deviation of the Raman laser phase difference for an integration
  time equal to the interferometer pulse separation $T$. The errors
shown are the standard error from partitioning the data into 10 sets.} 
  \label{fig:allan_dev_m2}
\end{figure}

\subsection{Vibrations}\label{subsec:vibration_noise}
Another source of phase noise arises from vibrations of the
retro-reflecting mirror. Since one of the Raman beams is reflected, the Raman phase depends upon the position of the atoms relative
to the mirror. This defines a frame of
reference for the position of the atoms in which their acceleration is
measured. Any
random motion of the mirror, for instance from mechanical vibrations,
introduces a random component to the laser phase~\cite{Vigue2006}. If
the mirror moves by an amount $\delta z$ and we assume that
$\textbf{k}_2$ is the reflected beam then the Raman phase changes by
an amount $\delta \phi = 2 \textbf{k}_2 \delta z \approx - \keff \delta
z$. Therefore, in terms of the position power spectral density $S_z
(\omega)$, the interferometer phase power spectral density is
$k_\text{eff}^2 |H(\omega)|^2 S_z(\omega)$. To see how this relates to
acceleration noise, we first express $z(t)$ in terms of its Fourier
spectrum. After differentiating twice, it is straightforward to show
that the acceleration power spectral density $S_a(\omega)$ is given by
\begin{equation}
  S_a(\omega) = \omega^4 S_z(\omega) 
\end{equation}
If $|H_a(\omega)|^2$ is the acceleration transfer function, then
$|H_a(\omega)|^2 S_a(\omega) = |H(\omega)|^2 S_\phi(\omega)$ which
results in
%An acceleration of
%the mirror modifies the laser phase as follows
%\begin{equation}
%  \frac{\mathrm{d}^2 \Phi(t)}{\mathrm{d} t^2} = \keff.\textbf{a}(t)
%  \label{eq:phase_acc}
%\end{equation}
%and the sensitivity to accelerations \(g_a\) is given by
%\begin{equation}
%  \frac{1}{k_\text{eff}} \frac{\mathrm{d}^2 g_a(t)}{\mathrm{d} t^2} =
%  g(t) \\
%  \label{eq:acc_sens}
%\end{equation}
%Assuming that the pulse time \(\tau\) is much shorter than the
%interferometer pulse separation, \(T\), the acceleration sensitivity
%function is approximated by
%\begin{equation}
%  g_a(t) = \begin{cases}
% -1 & - T < t < 0\\
% 1 & 0 < t < T
%  \end{cases}
%  \label{eq:acc_sens_approx}
%\end{equation}
\begin{equation}
  |H_a(\omega)|^2 = \frac{k_\text{eff}^2}{\omega^4}|H(\omega)|^2 
  \label{eq:acc_transfer}
\end{equation}
In the low frequency limit \(\omega \ll \Omega\), this simplifies two
\begin{equation}
  |H_a(\omega)|^2 = \frac{16 k_\text{eff}^2}{\omega^4}
  \sin\left(\frac{\omega T}{2}\right)^4
  \label{eq:acc_tf_low}
\end{equation}
\FigureRef{fig:transfer_func_acc}
shows the acceleration transfer function for a pulse separation time
of \sivalue{25}{\ms}. There is a significant suppression of the gain
at frequencies greater than $1/(2T)$ and the interferometer is most
sensitive to low-frequency vibrations. This can be understood by the
fact that the phase shift arising from
low-frequency vibrations will not average out over the duration of the
interrogation time $2T$.  
\begin{figure}[htpb]
  \centering
  \includegraphics[width=0.7\textwidth]{transfer_func_acc}
  \caption[Acceleration noise transfer function.]{Acceleration noise
  transfer function. The parameters used here are the same as those
previously defined for~\FigureRef{fig:transfer_function}.}
  \label{fig:transfer_func_acc}
\end{figure}
\subsubsection{Measuring the Acceleration Noise Power Spectral Density}
A sensitive measurement of acceleration requires that the shot-to-shot
fluctuations arising from low-frequency vibrations of the retro-reflecting
mirror are sufficiently small. In a lab environment, low-frequency
vibrations are pervasive so it is necessary to make use of some form
of isolation to reduce their effect on the interferometer phase.
We employ two methods that both aim to reduce the
vibrations that couple into the apparatus from the ground through the
optical table. Firstly, the vacuum
chamber is mounted with a layer of Sorbothane placed between it and the optical
table to damp the vibrations. The optical table is passively isolated
from the ground using a pneumatic suspension system between the optical table and its
supporting legs. This pneumatic suspension system can be easily
switched off, which makes it possible to consider the effect on the
interferometer phase 
from comparatively low and high levels of vibration.
\par\noindent
\FigureRef{fig:vibration_trace} shows a measurement of the MEMS
accelerometer signal over a period of \sivalue{1}{\s}. The blue curve,
which was obtained when the pneumatic suspension was disabled, shows
significantly higher levels of vibration noise. The orange curve was
obtained with the additional isolation and accordingly shows a quieter
signal. An oscillation of the
signal at \sivalue{50}{\Hz} is also apparent in both cases. This is
characteristic of interference from other electrical sources and has
since been suppressed by isolating the power source of the MEMS
accelerometer. Since this is outside the bandwidth of the
interferometer (but not of the MEMS), its contribution to the
interferometer phase is also suppressed. \FigureRef{fig:vibration_psd}
shows the interferometer power spectral spectral density for a pulse
separation time of $T = $ \sivalue{25}{\ms} with and
without the pneumatic suspension. The suspension acts as a
low-pass filter and reduces the power within the
10-\sivalue{200}{\hertz} bandwidth, which is aliased into the
sub-\sivalue{10}{\hertz} 
frequency band. 
\begin{figure}[htpb!]
  \centering
  \subfloat[][]{\includegraphics[width=0.5\textwidth]{vibration_trace}\label{fig:vibration_trace}}
  \subfloat[][]{\includegraphics[width=0.5\textwidth]{vibration_spectrum_tf.pdf}\label{fig:vibration_psd}}
  \caption[Interferometer power spectral density due to vibration
  noise.]{\textbf{a} Acceleration of the retro-reflecting mirror
    measured by the MEMS accelerometer. \textbf{b} Interferometer 
  phase power spectral density due to vibrations for an interferometer
pulse separation of $T = $ \sivalue{25}{\ms}. In both plots, the blue
curves were taken without further isolation using pneumatic suspension
(see text). The orange curves were obtained with this enabled.} 
  \label{fig:vibration_spectrum}
\end{figure}
\par\noindent
We can consider the effect of vibration noise on interferometer phase
using the same technique that was previously used for laser phase noise
in~\SectionRef{subsec:laser_phase}. The signal from the MEMS
accelerometer was measured for \sivalue{5}{\minute} and partitioned
into subsets of duration $T$. These were doubly integrated to find the
change in position of the mirror $\delta z$, from which the
interferometer phase shift $\Delta \Phi$ is calculated. As before, the
blue curve shows the expected fluctuations without the additional
pneumatic suspension whereas the orange curve includes this. 
\FigureRef{fig:vibration_shot_fluct} shows the mean shot-to-shot
fluctuation as a function as the interferometer pulse separation. It
is clear that the interferometer phase is greatly affected by the
external vibrations. At $T = $ \sivalue{40}{\ms}, the
interferometer phase fluctuation differs by a factor of 4 under high and low levels of
vibration noise. Conversely, the shot-to-shot fluctuations
are comparable at separation times below \sivalue{10}{\ms}. 
For a pulse separation of \(T =\) \sivalue{25}{\ms}, the expected value is $\sigma_\Phi =$
\sivalue{2.5}{\radian} and \sivalue{0.5}{\radian} for the
respective cases of high and low vibration noise. In both instances,
this is significantly larger than the other sources of noise. This
highlights the importance of isolating the apparatus from vibrations.
Indeed, the vibration noise in our lab is likely much lower than on a
moving vehicle such as a submarine. 
A sensitive, mobile sensor will certainly require a more sophisticated
method of isolating the apparatus from external vibrations. For
instance, active
vibration isolation systems can be engineered to have a low natural
frequency, on the order of
\sivalue{0.01}{\milli\hertz}\cite{Zhou2012}. An active isolator would
substantially suppress vibration noise up to the interferometer phase
bandwidth of $1/(2T)$ and enhance the sensitivity to the atoms'
acceleration.
\begin{figure}[htpb]
  \centering
  \includegraphics[width=0.7\linewidth]{vibration_shot_fluct.pdf}
  \caption{Shot-to-shot fluctuations of the Interferometer phase due
  to vibrations. The standard error from 5 sets of data is also
indicated. As in~\FigureRef{fig:vibration_spectrum}, the blue curve
shows the case of worse vibration isolation than in the orange curve.}
  \label{fig:vibration_shot_fluct}
\end{figure}
%\subsection{Other Sources of Phase Noise}
%Aside from laser phase noise and vibrations, there are additional effects
%which result in interferometer phase noise. These have not been
%characterised in this experiment as their magnitude is typically much
%smaller than the present vibration phase noise. 
%\begin{itemize}
%  \item \underline{Magnetic field gradients}. The \(\Delta_m = 0\)
%    clock transition has a second-order Zeeman shift of
%    \sivalue{575}{\Hz\gauss\tothe{-2}}. A gradient of magnetic field
%    across the area traversed by the atoms introduces a propagation
%    phase which does not cancel. 
%  \item \underline{Laser intensity noise}.
%\end{itemize}
\subsection{Summary of Identified Noise Sources}\label{subsec:noise_sources}
To conclude this discussion, it is worth summarising the
sources of noise that have been presented. 
\TableRef{tab:noise_sources} shows the expected contributions from
each of these sources to the shot-to-shot fluctuation of the
interferometer phase and the corresponding acceleration uncertainty
for an interferometer pulse separation time of $T =$
\sivalue{25}{\ms}. By far, the largest
contribution to the acceleration uncertainty comes from vibrations of
the retro-reflecting mirror. As mentioned above, an improvement to the
vibration isolation system would certainly reduce this. Another method
for reducing the uncertainty induced by vibrations is discussed
in~\SectionRef{subsec:vibration_sensitivity}. This uses the
MEMS accelerometer to measure the vibration of the mirror. The
vibration induced phase is then subtracted
from the measured interferometer phase.
\par\noindent
The noise sources presented are by no means the only effects which
limit the sensitivity of the interferometer to accelerations. For
instance, a magnetic field gradient over the volume occupied by the
atoms induces a phase shift due to the second-order Zeeman shift of
the Rubidium-87 clock transition. 
Other noise sources, such as magnetic field gradients, laser intensity
noise and rotations have not yet been characterised. Higher-order
phase shifts due to inertial effects, such as the Coriolis force and
gravity gradients, have been
studied elsewhere\cite{Bongs2006}.
\begin{table}[htpb!]
  \centering
  \begin{tabular}{ccc}
    \toprule
    Noise Source  & $\sigma_{\Phi}$  (\sivalue{}{\milli\radian})  &\(\sigma_a\)
    (\sivalue{}{\micro\meter\per\s\squared}) \\
    \midrule
    Projection noise (\num{3e6} atoms) & 1 & 0.1 \\
    Technical detector noise & 2 & 0.2 \\
    Laser phase noise  & 10 & 1 \\
    Vibrations & 500 & 50 \\
    \bottomrule
  \end{tabular}
  \caption[Comparison of known noise sources.]{Comparison of
    identified noise sources and their effects on
  acceleration measurements. These values are estimated assuming a
separation between pulses of \(T = \) \sivalue{25}{\ms} and a \(\pi/2\)
pulse time of \(\tau = \) \sivalue{20}{\micro\s}. The projection noise
assumes perfect fringe contrast.}
  \label{tab:noise_sources}
\end{table}
\section{Measuring Accelerations}\label{sec:atomint_accelerations}
This section presents the technique used to characterise the atom interferometer
and its sensitivity to accelerations. It begins with a calibration of
the fringe pattern in~\SectionRef{sec:fringe_cal}. Following this, a
method for subtracting the vibration-induced interferometer phase is given
in~\SectionRef{subsec:vibration_sensitivity}. Finally, this section concludes with
a measurement of the Allan deviation to examine the stability of the
interferometer signal in~\SectionRef{subsec:stability}.
\subsection{Fringe Calibration}\label{sec:fringe_cal}
The presence of background atoms and inhomogeneities in the Rabi
frequency across the atom cloud mean that we do not observe a perfect
interferometer fringe contrast. In order to infer an interferometer
phase $\Phi$ from the transition probability, the interferometer fringe
pattern must be calibrated to determine the fringe contrast and mean
transition probability~\cite{Peters2001}. 
In addition to the acceleration of the atoms, the interferometer phase
is also sensitive to the Raman phase of each laser pulse. Recalling
from~\SectionRef{subsec:laser_phase}, this is given by $-(\phi_0)_1
+2(\phi_0)_2 - (\phi_0)_3$, where $-(\phi_0)_j$ is the phase of the
beat-note between the two Raman frequencies during the $j$-th pulse.
The interferometer fringe can be obtained by varying the phase of one
of these pulses.
\par\noindent
\FigureRef{fig:fringe_examp} shows an interference fringe obtained by
varying \((\phi_0)_2\) for an interferometer pulse separation time of
$T=$ \sivalue{25}{\ms}. In this instance, the contrast is $C=$
\num{0.042\pm0.003} and the mean
transition probability is $\textnormal{P}_0 =$
 \num{0.39\pm0.001}. The fringe contrast is far smaller than the ideal
 case of $C = 1$. Since the acceleration uncertainty $\sigma_a$ is inversely
 proportional to the fringe contrast, we expect a quantum projection
 noise-limited value of $\frac{1}{C} $
 \sivalue{0.1}{\micro\m\per\second\squared} $\approx$
 \sivalue{2.5}{\micro\m\per\second\squared}. It should also be noted
 that the characteristic $\sin(\phi)$ dependence of the quantum
 projection noise is not observed in the fringe pattern, since the
 vibration-induced phase noise is dominant\nocite{Sugarbaker2013}.
\begin{figure}[htpb!]
  \centering
  \includegraphics[width=0.7\textwidth]{fringe_examp}
  \caption[Interference fringe for \(T = \)\sivalue{25}{\ms}.]{Interference fringe obtained by varying the phase
    difference of the two Raman lasers during the middle \(\pi\) pulse
    for a pulse separation time of \(T = \)\sivalue{25}{\ms}. The orange
curve is a non-linear least squares fit to the data, giving a contrast
of $C=$ \num{0.042\pm0.003} and a mean
transition probability of $\textnormal{P}_0 =$
 \num{0.39\pm0.001}.}
 \label{fig:fringe_examp}
\end{figure}
\subsection{Correcting for Vibration Noise}\label{subsec:vibration_sensitivity}
Vibrations of the retro-reflecting mirror are a significant source of
phase noise, which limits the sensitivity of the interferometer to
accelerations. This is particularly apparent when the vibration noise
induces a phase shift of greater than 2\(\pi\) radians. If the
interferometer signal spans multiple fringes, it is not possible to
accurately determine acceleration from the interferometer phase.
\par\noindent
One method to filter the effects of vibration noise uses the MEMS
accelerometer to measure the position of the retro-reflecting mirror
during the interferometer pulse sequence~\cite{Merlet2009}. The
integral of the voltage across the output of the MEMS is proportional
to the velocity of the mirror.
Recalling~\EquationRef{eq:phase_contrib}, the phase shift due to
vibrations is therefore given by the following
\begin{equation}
  \Phi_\textnormal{vib} =  k_\text{eff} K\int_{-T}^{T} g(t) V_m(t)
  \mathrm{d}t
  \label{eq:phase_vib}
\end{equation}
where \(V_m(t)\) is the integrated MEMS output voltage, \(K = \)
\sivalue{0.793}{\m\s\tothe{-2}\per\V} is the
scaling factor from voltage to acceleration and \(g(t)\) is 
the sensitivity function, defined
in~\EquationRef{eq:sensitivity_interferometer}. For simplicity, it has been
assumed that the change in position of the mirror during the
interferometer pulses is negligible. To correct for the vibration
induced phase, the measured transition probability is
fit to the fringe function
\begin{equation}
  \text{P}_{\ket{F=2}} = \text{P}_0 + \frac{C}{2}\cos(\kappa
  \widetilde{V}_\textnormal{vib} + \Phi_\text{off})
  \label{eq:fringe_fit_vibration}
\end{equation}
where $\widetilde{V}_\text{vib}$ is the integral in
\EquationRef{eq:phase_vib} such that $\kappa \widetilde{V}_\text{vib}
= \Phi_\text{vib}$ The parameters $\text{P}_0$, $C$, $\kappa$ and $\Phi_\text{off}$ are free
parameters, to be determined from the fit. The scaling factor $\kappa$ is known \textit{a priori},
but it is left free to verify that the fitted value is in close
agreement with the expected value. In the absence of other noise sources,
$\Phi_\text{off}$ represents the acceleration-induced interferometer
phase\footnote{Of course, this is still influenced by other noise
  sources, such as laser phase noise and detector noise, which the
MEMS accelerometer is not sensitive to.}. The
interferometer phase is set such that $\Phi_\text{off} = \pi/2$ and
the interferometer is operates close to the
mid-point of a fringe. For vibrations which induce a phase shift less
than $\pi\,$\sivalue{}{\radian}, the change in the
transition probability is approximately linear in a change in
$\Phi_\text{vib}$. The
phase error $\delta \Phi$ that results from other noise sources can be estimated from the
residuals of the fit to \EquationRef{eq:fringe_fit_vibration}.
Denoting the residuals as $\delta \text{P}$, the phase error $\delta
\Phi$ is given by 
\begin{equation}
\delta \Phi = \cos^{-1}\left(\frac{2 \delta \text{P}}{C}\right)
\end{equation}
\FigureRef{fig:low_vib} plots the correlation between the
vibration-induced phase measured using the MEMS accelerometer
and the transition probability in the presence of small vibration
noise. The orange line is a best-fit
to~\EquationRef{eq:fringe_fit_vibration} and it is clear that the
measured transition probability remains within one half of a fringe. A prior calibration of the
fringe contrast gave a value of $C = $ \sivalue{0.05}. It is helpful
to compare the distribution of the measured interferometer phase with
the residual phase error $\delta \Phi$ to understand the level to
which the vibration noise is rejected from the interferometer phase
noise. \FigureRef{fig:low_vib_hist_mems_atom} shows a histogram of the
deviation of the measured acceleration (determined from the
interferometer phase). The blue values are the values obtained from
the transition probabilities and the green are obtained from the
residual phase error. The dashed lines indicate fitted Normal
distributions. The respective standard deviations of the acceleration
are $\sigma_a^{(i)} =$ \sivalue{55}{\um \s\tothe{-2}} and
$\sigma_a^{(v)} = $ \sivalue{27}{\um \s\tothe{-2}}, giving a vibration
noise rejection factor of 2.04. If the vibration-induced phase noise
is perfectly rejected, we expect to reach an uncertainty on the order
of the laser phase noise $\sigma_a =$ \sivalue{1}{\um \s\tothe{-2}}.
The fact that the uncertainty is much greater than this indicates that
this technique is not able to fully reject the vibration noise. This
can be understood by the fact that the non-linear least-squares method
assumes that there is no error in the independent variable
$\widetilde{V}_\text{vib}$. The MEMS accelerometer is specified to
have an intrinsic noise level of < \sivalue{70}{\um\s\tothe{-2}} in the
0-\sivalue{100}{\Hz} band. It is likely that there is additional error
due to this which is limiting the vibration
rejection.\nocite{Macdonald1992} 
%A weighted least-squares fit is
%able to account for errors in both variables~\cite{Macdonald1992}. This requires an
%accurate estimate of the weights for each measurement of
%$\tilde{V}_\textnormal{vib}$ and $\text{P}_{\ket{F=2}}$ to avoid inaccurate
%parameter estimates.
\begin{figure}[htpb!]
  \centering
  %\subfloat[][]{\scalebox{0.5}{\includegraphics{high_vib}}\label{fig:high_vib}}
    \subfloat[][]{\scalebox{0.3}{\includegraphics{low_vib}}\label{fig:low_vib}}
    \subfloat[][]{\scalebox{0.3}{\includegraphics{low_vib_hist_atom_res}}\label{fig:low_vib_hist_atom_res}}
  \caption[MEMS/Interferometer correlation in a
    low vibration
    environment.]{\textbf{(a)} Correlation of the vibration-induced interferometer
    phase $\Phi_\textnormal{vib}$ with the measured transition
    probability in the presence of low vibration noise. \textbf{(b)}
    is a histogram of the acceleration about its mean value. The blue
    points are obtained using the measured probabilities and the green
    points are obtained from the residuals of the fit shown in
    \textbf{(a)}. The dashed lines are fitted Normal distributions.}
  \label{fig:vib_comparison}
\end{figure}
\par\noindent
This method assumes that the phase shift due to vibrations is small
and becomes inaccurate when the vibration-induced phase spans multiple
fringes. In this situation, we can make use of another technique
described in~\cite{Merlet2009}. This involves binning the data into
sets of $N_\text{samp}$ = 10 points and fitting each of these to
extract a corresponding $\delta \phi$. The residual phase noise is
then obtained from the rms of these. \FigureRef{fig:high_vib} shows
the transition probabilities correlated with the vibration-induced
phase in the case of high vibration noise. The red lines are
individual fits from $N_\text{samp}$ points. The rms of the phase
errors gives an
estimate on the acceleration uncertainty of
\sivalue{1.6e-4}{\um\s\tothe{-2}}.  
%The suppression of the vibration noise can be seen in the distribution
%of the estimated acceleration. \FigureRef{fig:vibration_hists} shows
%histograms of the acceleration measured by the MEMS and the
%interferometer in the low vibration instance
%of~\FigureRef{fig:low_vib}. \FigureRef{fig:low_vib_hist_mems_atom} compares the The noise in the MEMS signal is larger than in the
%interferometer -- their respective standard deviations are
%$\sigma_a^{(m)} =$ \sivalue{66.8}{\um \s\tothe{-2}} and
%$\sigma_a^{(i)} =$ \sivalue{20.8}{\um \s\tothe{-2}}. The MEMS
%accelerometer has a higher acceleration bandwidth (\sivalue{20}{\kHz}) than the
%interferometer (\sivalue{20}{\Hz}). Consequently, the interferometer
%is not sensitive to the high-frequency noise measured by the MEMS
%accelerometer. \FigureRef{fig:low_vib_hist_atom_res} compares the acceleration noise
%from the interferometer with the acceleration inferred from the fit
%residuals
%$\phi_\textnormal{res}$. The latter has a standard deviation of
%$\sigma_a^{(r)} = $ \sivalue{10.4}{\um \s\tothe{-2}}. This method is
%able to filter the effects of vibration from the interferometer
%signal. However, the non-linear least squares method assumes that
%there is no error in the independent variable
%$\tilde{V}_\textnormal{vib}$. This introduces a random
%error to $\phi_\textnormal{res}$ from the noise intrinsic to the MEMS
%accelerometer. 
\begin{figure}[htpb!]
  \centering
    \includegraphics{high_vib}
  \caption[MEMS accelerometer and transition probability in high
  vibration noise]{MEMS accelerometer and transition probability in high
  vibration noise. The red lines are least-squares sinusoidal fits to
subsets of $N_\text{samp} = 10$ points. The vibration phase
$\Phi_\text{vib}$ and transition probability are plotted relative to
their mean values.}
  \label{fig:vibration_hists}
\end{figure}

\subsection{Signal Stability}\label{subsec:stability}	
The stability of the interferometer's sensitivity to accelerations can
be determined from the Allan deviation. A comparison of the
Allan deviation in the presence of high and low vibrations is shown
in~\FigureRef{fig:adev_comparison}. In both instances, the sensitivity
to accelerations is improved by correcting for the vibration-induced
phase. \FigureRef{fig:high_vib_adev} shows the Allan deviation
measured under high vibrations (without floating the optical table).
The Allan deviation has a minimum value
of around \sivalue{3e-6}{\m\s\tothe{-2}} after integrating for
\sivalue{35}{\s}. This is a bias instability, i.e. fluctuations in
the bias. This value is obtained by subtracting the vibration
phase, estimated from the MEMS accelerometer, from the interferometer
phase. There is additional noise in the MEMS
accelerometer which leads to this bias instability. At short
integration times, the Allan deviation has a $\tau^{-1}$
dependence. It is likely that this arises from additional low
frequency noise from the MEMS signal~\cite{Meunier2015, Fang2016}. This is
well-correlated between successive measurements, where the dead time
between them means that full dynamics of the system are not captured. In the context of
atomic clocks, this is referred to as the Dick effect~\cite{Dick1990}.  
\par\noindent
\FigureRef{fig:low_vib_adev} shows the Allan devation in the case of
smaller vibrations. In contrast to the previous case, the signal
remains stable for a longer period of time. The Allan deviation
is proportional to \(\tau^{-1/2}\), which is characteristic of
uncorrelated white phase noise. At longer integration times, the
small number of samples introduces a large uncertainty on the Allan
deviation. For times up to \sivalue{100}{\s}, 
the sensitivity of the signal is not limited by bias
instability. 
\begin{figure}[htpb!]
  \centering
  \subfloat[][]{\scalebox{0.3}{\includegraphics{high_vib_adev}}\label{fig:high_vib_adev}}
    \subfloat[][]{\scalebox{0.3}{\includegraphics{low_vib_adev}}\label{fig:low_vib_adev}}
  \caption[Comparison of Allan deviation in a high and
    low vibration
  environment.]{Allan deviation of the estimated acceleration using
    the interferometer signal. In each plot the orange curve
    represents the sensitivity to accelerations without subtracting
    the vibration induced phase, and the blue curve shows the
    sensitivity after subtracting this. 
    As in~\FigureRef{fig:vib_comparison}, \textbf{(a)} shows
    the sensitivity in a high vibration environment, and
    \textbf{(b)} shows it after reducing the level of vibration.}
  \label{fig:adev_comparison}
\end{figure}

\section{Conclusion}
This chapter has described the methods used to observe matter-wave
interference. It has presented a description of the Raman laser system
and the scheme used to infer the population in each internal state. 
The light pulses used to drive Raman
transitions were subsequently presented, to emphasise the effect of
a light shift and background atoms on the interferometer. A further
discussion of identified noise sources showed that vibrations are the
largest contributor. The interference between two states was
characterised
