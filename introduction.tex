\chapter{Introduction}\label{chap:intro}
\section{Light-Pulse Atom Interferometry}
The development of quantum theory to explain the wave-like behaviour
of matter has enabled significant advances in science and technology.
An early example of this is the atomic clock~\cite{ESSEN1955}, which makes use of the
interference between two atomic internal states to precisely measure
their transition frequency. This technology has led to a
re-definition of the second in terms of physical constants, as well as
a globally-adopted standard for timekeeping~\cite{Levine} that supports many aspects
of modern society. Early interferometers relied on driving microwave
frequency transitions because the contemporary laser technology did
not have the frequency stability or linewidths necessary to maintain coherence
between the atomic states. It was later
suggested that interference of matter-waves could be used to make
sensitive measurements of accelerations and
rotations~\cite{Clauser1988}. This sensitivity is increased using the large momentum
recoil of optical frequency transitions to make an interferometer
using successive pulses of light~\cite{Borde1989}.\par\noindent
Kasevich and Chu demonstrated inertial force sensing using atom
interferometry with a measurement of gravitational acceleration~\cite{Kasevich1991,Kasevich1992}. 
Subsequent work showed that atom interferometers were also able to
measure rotations~\cite{Durfee2006,Dubetsky2006}. As inertial sensors, these
experiments showed comparable short-term sensitivities to their
classical counterparts. By contrast, their long-term stability was
noteworthy. This is due to the frequency-stability of the atomic
transition and in the lasers used to stimulate it. Again, this is
analogous to atomic clock systems which are able to reach very low
fractional uncertainties by integrating over long
timescales~\cite{Borde2002}. 
\section{Application to Inertial Navigation}
Mobile applications of atom interferometry as inertial sensors have
demonstrated both high sensitivity and long-term stability outside of
laboratory environments~\cite{Barrett2013}. Experiments during the
parabolic flight of an aircraft have also demonstrated measurements of
acceleration around \sivalue{0}{\g} and
\sivalue{1}{\g}~\cite{Nyman2006a,Geiger2011a}. 
The ability to precisely measure a range of accelerations over long durations
is of great practical interest for inertial
navigation systems. These measure the position of a
moving body using successive measurements of its acceleration and
rotation, known as dead reckoning. The accuracy of this method depends
on the intrinsic noise of each sensor, as well as their bias. If the
noise is known \textit{a priori}, then a statistical method such as a
Kalman filter~\cite{Faruqi2000} can be used to produce a more accurate estimate of the
position. However, the error from a bias tends to dominate over long
timescales. For instance, the position
error from a constant acceleration bias grows quadratically with time.
This problem is compounded by the fact that the bias in an inertial
sensor drifts over time due to physical effects such as temperature
variation and mechanical strain. This bias instability means that
inertial navigation systems require regular re-calibration and an
external method of correcting for the position error such as
\ac{gnss}~\cite{Wen2016}. 
\par\noindent
One drawback of atom interferometers is the interrogation time needed
between each laser pulse. This leads to a sensitivity bandwidth of the order of
\sivalue{10}{\Hz}. It should be noted that acceleration bandwidths as
high as 
\sivalue{60}{\Hz}~\cite{Rakholia2014a} and even
\sivalue{10}{\kHz}~\cite{Biedermann2017} using short, high-power
pulses
have been reported, but these are at the cost of a reduction in sensitivity.
A further issue is the measurement dead time as a result of the time
required to prepare a cold ensemble of atoms. During this preparation time, the system
does not respond to any acceleration or rotation acting on the atoms.
It has been suggested that this shortcoming can be counteracted by
combining the interferometer with a mechanical accelerometer~\cite{JEKELI2005} or by
interleaving multiple cold atom interferometers. The latter of these
has been demonstrated in a gyroscope~\cite{Dutta2016}, but is
technically more challenging. 
\par\noindent
Hybridising an atom interferometer with a classical
sensor results in a composite system which aims for both long-term
stability and high bandwidth. Indeed, a secondary
measurement is needed to determine the interferometer fringe order and hence obtain
an absolute value of acceleration. This is of particular importance
for inertial navigation, where the range of acceleration is far
greater than the fringe period~\cite{Merlet2009}. This auxiliary sensor can also
improve the sensitivity of the interferometer in high vibration
environments~\cite{Lautier2014}, which is a significant source of phase
noise. Measuring these vibrations provides a method of filtering their
effect from the interferometer signal. 
\nocite{Dimopoulos2008}
\section{Aims}
This aim of this project was to investigate the use of atom
interferometry in the context of inertial navigation by measuring
horizontal accelerations. Gravitational acceleration induces motion
transverse to the laser wavefront. Together with the requirements of
high acceleration sensitivity and minimal dead-time, this influenced the technical
aspects of this experiment.  
\section{Note on Units}

In subsequent chapters of this thesis, various non-SI units are used
for convenience. The commonly used ones are defined here:
\begin{itemize}
  \item \sivalue{1}{\gauss} \(= \sivalue{1e-4}{\tesla}\) (Gauss - cgs
    unit of magnetic flux density)
    \item g = \sivalue{9.80665}{\metre\per\second\squared}
      (international standard of gravitational acceleration, measured
      at $45^\circ$ latitude and sea level~\cite{accelStandard})
    \item $I_\textnormal{sat}$ = \sivalue{1.38}{\mW\per\cm\tothe{2}}
      (saturation intensity for the \trans{2,2}{3,3} cycling transition in
      Rubidium-87)
    \item $\Gamma$ = $2\pi \times$ \sivalue{6.065}{\MHz} (natural
      linewidth of the above transition)
  \end{itemize}
\section{Structure of this Thesis}
This thesis is structured as follows:
\begin{itemize}
  \item \textbf{Chapter 2} presents a theoretical introduction to
    matter-wave interferometry and the relevant atomic structure of
    Rubidium-87.
  \item \textbf{Chapter 3} describes the software used to control the
    experiment and acquire data.
  \item \textbf{Chapter 4} outlines the preliminary cooling and
    trapping of Rubidium-87 in a \ac{mot}.
  \item \textbf{Chapter 5} discusses the techniques used to further
    cool the atoms and prepare an ensemble in a suitable single internal state,
  \item \textbf{Chapter 6} gives a detailed description of the
    in-vacuum optical system for driving Raman transitions.
  \item \textbf{Chapter 7} characterises the interferometer's
    sensitivity to accelerations.
  \item \textbf{Chapter 8} summarises the conclusions of this thesis and outlines
    further work towards improving the accelerometer.
\end{itemize}


