\chapter{Introduction}\label{chap:intro}
\section{Technology and Quantum Mechanics}
Throughout history, scientific progress has paved
the way towards great technological and societal advances. A
(relatively) modern
example of this is the development of quantum mechanics,
around the start of the 20\textsuperscript{th} century. This led to
vast improvements in the understanding of the physical world at the
microscopic scale. Eventually, the application of quantum theories of
matter and light led
to technological discoveries which have hugely impacted upon
subsequent scientific research, as well as our general lifestyle. These benefits are not always immediately obvious. For
instance, around the time that the laser was
discovered~\cite{Schawlow1958}, it
was believed that a coherent source of light was of little practical
use
outside of spectroscopy. Of course, the widespread use of commercial laser systems in
medicine and telecommunications show that this is not the case.
Another such example is the development of semiconductor electronics, which started with the transistor and has revolutionised
modern life with the plethora of devices seen today. 
\par\noindent
Broadly speaking, these technologies rely on quantum phenomena that
exist within
bulk matter, i.e. macroscopic ensembles of quantum objects. 
These do not exhibit the microscopic quantum phenomena such as
entanglement and interference. They are open systems, which interact strongly with their
environment causing a loss of coherence across the ensemble. For
this reason, there has been a
growth in research that has focused on the control of
individual quantum systems, such as atoms, molecules and photons.
Experimentally, these can be well-isolated from their environment so
that the coherence between quantum states affects the dynamics of the
system. Technologies aimed at controlling quantum systems have
progressed to the point where it is now possible to engineer robust
devices that make use of quantum effects. This is well-known in the
context of quantum computing and quantum cryptography, which offer
many benefits over their classical counterparts. 
\par\noindent
The field of metrology has also shown great increases in measurement
precision using quantum systems. The archetypal example of this is the
atomic clock~\cite{ESSEN1955}, which uses the interference between two internal states
of an atom such as Caesium-13 to
measure their transition frequency. Owing to their frequency
stability, atomic clocks are now used to implicitly define the
second in terms of physical constants and provide a continuous,
precise time scale~\cite{Levine}.

\section{Inertial Navigation}
The ability to measure inertial forces, i.e. accelerations and
rotations, is of
great practical interest. This is strongly motivated by the need for
inertial navigation systems. In general, these make use on on-board
sensors to determine its position using the process of dead reckoning.
The acceleration and rotation are periodically measured and the
position of the object relative to its initial position is determined by
integrating these forces. The accuracy of this method depends upon the intrinsic noise and bias of these devices. These
lead to a position error which grows in time. For instance, the
effects of white noise are discussed in~\cite{Wen2016}. In the context
of this project, the effects of a sensor's stability over time is
discussed more quantitatively using the Allan variance
in~\SectionRef{subsec:allan_variance}.
\par\noindent
Generally, the
position error due to sensor noise can be reduced through statistical
techniques such as Kalman filtering. This produces an un-biased estimate
of the updated position by taking the noise of each sensor into
account. The position error due to a bias must be addressed
separately. A constant acceleration bias, $\epsilon$ leads to a
position error that grows quadratically in time. An initial
calibration of the sensor can remove this bias, but cannot account for
drifts over time. This phenemenon, often termed bias instability, is
the result of effects such as temperature variations and mechanical
strains. This leads to a characteristic time beyond which the
measurement uncertainty is not reduced by averaging the signal. This
problem can be alleviated by regularly correcting for the position
error using an external positioning system, such as \ac{gnss}, and
re-calibrating the sensors.
\subsection{Improving Long-Term Stability with Atom Interferometry}
Experiments over the past few decades have highlighted the prospect of
using cold atoms for precise inertial sensing. Once cooled to the
\sivalue{}{\mu\K} regime, the thermal velocity of an atomic ensemble
is small enough that they can be controlled for durations up to the order
of seconds. Early experiments by Kasevich and Chu into matter-wave
interferometry of Sodium atoms demonstrated the 
It has been suggested that the problem of bias instability can be
overcome by using cold atoms to measure inertial forces.  
For instance, there exist commerically available \ac{mems} devices
which have a noise level of $\sim \sivalue{5}{\mu\g}$ in the
0-\sivalue{100}{\Hertz} bandwidth. Since the noise between successive
measurements is uncorrelated, the position error  After many successive measurem  
\section{Thesis Overview}
This thesis is structured as follows. A theoretical introduction to
matter-wave interferometry and the relevant atomic structure of
\ac{rb87} is presented in~\ChapterRef{chap:theory}. This is followed
by a description of the software that was developed to control the experiment and
acquire data in~\ChapterRef{chap:compinterface}. The subsequent chapters
focus on various experimental aspects of the project. The preliminary
cooling and trapping of \ac{rb87} is presented
in~\ChapterRef{chap:mot}. The techniques used to cool the atoms
further and prepare an ensemble in the same internal state are
explained in~\ChapterRef{chap:atom_prep}. This is followed by the
design and characterisation of an in-vacuum optical system for driving
Raman transitions in~\ChapterRef{chap:raman_optics}. Finally, a
characterisation of acceleration-sensitive interference is given
in~\ChapterRef{chap:atom_int}.



