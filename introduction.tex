\chapter{Introduction}\label{chap:intro}
%\section{Technology and Quantum Mechanics}
%Throughout history, scientific progress has paved
%the way towards great technological and societal advances. A
%(relatively) modern
%example of this is the development of quantum mechanics,
%around the start of the 20\textsuperscript{th} century. This led to
%vast improvements in the understanding of the physical world at the
%microscopic scale. Eventually, the application of quantum theories of
%matter and light led
%to technological discoveries which have hugely impacted upon
%subsequent scientific research, as well as our general lifestyle. These benefits are not always immediately obvious. For
%instance, around the time that the laser was
%discovered~\cite{Schawlow1958}, it
%was believed that a coherent source of light was of little practical
%use
%outside of spectroscopy. Of course, the widespread use of commercial laser systems in
%medicine and telecommunications show that this is not the case.
%Another such example is the development of semiconductor electronics, which started with the transistor and has revolutionised
%modern life with the plethora of devices seen today. 
%\par\noindent
%Broadly speaking, these technologies rely on quantum phenomena that
%exist within
%bulk matter, i.e. macroscopic ensembles of quantum objects. 
%These do not exhibit the microscopic quantum phenomena such as
%entanglement and interference. They are open systems, which interact strongly with their
%environment causing a loss of coherence across the ensemble. For
%this reason, there has been a
%growth in research that has focused on the control of
%individual quantum systems, such as atoms, molecules and photons.
%Experimentally, these can be well-isolated from their environment so
%that the coherence between quantum states affects the dynamics of the
%system. Technologies aimed at controlling quantum systems have
%progressed to the point where it is now possible to engineer robust
%devices that make use of quantum effects. This is well-known in the
%context of quantum computing and quantum cryptography, which offer
%many benefits over their classical counterparts. 
%\par\noindent
%The field of metrology has also shown great increases in measurement
%precision using quantum systems. The archetypal example of this is the
%atomic clock~\cite{ESSEN1955}, which uses the interference between two internal states
%of an atom such as Caesium-133 to
%measure their transition frequency. Owing to their frequency
%stability, atomic clocks are now used to implicitly define the
%second in terms of physical constants and provide a continuous,
%precise time scale~\cite{Levine}.
\section{Light-Pulse Atom Interferometry}
The development of quantum theory to explain the wave-like behaviour
of matter has enabled significant advances in science and technology.
An early example of this is the atomic clock~\cite{ESSEN1955}, which makes use of the
interference between two atomic internal states to precisely measure
their transition frequency. This technology has led to a
re-definition of the second in terms of physical constants, as well as
a globally-adopted standard for timekeeping~\cite{Levine} that supports many aspects
of modern society. Early interferometers relied on driving microwave
frequency transitions because the contemporary laser technology did
not have the frequency stability or linewidths necessary to maintain coherence
between the atomic states. It was later
suggested that interference of matter-waves could be used to make
sensitive measurements of accelerations and
rotations~\cite{Clauser1988}. This sensitivity is increased using the large momentum
recoil of optical frequency transitions to make an interferometer
using successive pulses of light~\cite{Borde1989}.\par\noindent
Kasevich and Chu demonstrated inertial force sensing using atom
interferometry with a measurement of gravitational acceleration~\cite{Kasevich1991,Kasevich1992}. 
Subsequent work showed that atom interferometers were also able to
measure rotations~\cite{Durfee2006,Dubetsky2006}. As inertial sensors, these
experiments showed comparable short-term sensitivities to their
classical counterparts. By contrast, their long-term stability was
noteworthy. This is due to the frequency-stability of the atomic
transition and in the lasers used to stimulate it. Again, this is
analogous to atomic clock systems which are able to reach very low
fractional uncertainties by integrating over long
timescales~\cite{Borde2002}. 
\section{Application to Inertial Navigation}
Mobile applications of atom interferometry as inertial sensors have
demonstrated both high sensitivity and long-term stability outside of
laboratory environments~\cite{Barrett2013}. Experiments during the
parabolic flight of an aircraft have also demonstrated measurements of
acceleration around \sivalue{0}{\g} and
\sivalue{1}{\g}~\cite{Nyman2006a,Geiger2011a}. 
The ability to precisely measure of a range of accelerations over long durations
is of great practical interest for inertial
navigation systems. These measure the position of a
moving body using successive measurements of its acceleration and
rotation, known as dead reckoning. The accuracy of this method depends
on the intrinsic noise of each sensor, as well as their bias. If the
noise is known \textit{a priori}, then a statistical method such as a
Kalman filter~\cite{Faruqi2000} can be used to produce a more accurate estimate of the
position. However, the error from a bias tends to dominate over long
timescales. For instance, the position
error from a constant acceleration bias grows quadratically with time.
This problem is compounded by the fact that the bias in an inertial
sensor drifts over time due to physical effects such as temperature
variation and mechanical strain. This bias instability means that
inertial navigation systems require regular re-calibration and an
external method of correcting for the position error such as
\ac{gnss}~\cite{Wen2016}. 
\par\noindent
One drawback of atom interferometers is the interrogation time needed
between each laser pulse. This leads to a sensitivity bandwidth of the order of
\sivalue{10}{\Hz}. It should be noted that acceleration bandwidths as
high as 
\sivalue{60}{\Hz}~\cite{Rakholia2014a} and even
\sivalue{10}{\kHz}~\cite{Biedermann2017} using short, high-power
pulses
have been reported, but these are at the cost of a reduction in sensitivity.
A further issue is the measurement dead time as a result of the time
required to prepare a cold ensemble of atoms. During this preparation time, the system
does not respond to any acceleration or rotation acting on the atoms.
It has been suggested that this shortcoming can be counteracted by
combining the interferometer with a mechanical accelerometer~\cite{JEKELI2005} or by
interleaving multiple cold atom interferometers. The latter of these
has been demonstrated in a gyroscope~\cite{Dutta2016}, but is
technically more challenging. 
\par\noindent
Hybridising an atom interferometer with a classical
sensor results in a composite system which aims for both long-term
stability and high bandwidth. Indeed, a secondary
measurement is needed to determine the interferometer fringe order and hence obtain
an absolute value of acceleration. This is of particular importance
for inertial navigation, where the range of acceleration is far
greater than the fringe period~\cite{Merlet2009}. This auxiliary sensor can also
improve the sensitivity of the interferometer in high vibration
environments~\cite{Lautier2014}, which is a significant source of phase
noise. Measuring these vibrations provides a method of filtering their
effect from the interferometer signal. 
%\section{Inertial Sensing with Atom Interferometry}
%Experiments over the past few decades have highlighted the prospect of
%using cold atoms for precise inertial sensing. Once cooled to the
%\sivalue{}{\micro\K} regime, the thermal velocity of an atomic ensemble
%is small enough that they can be controlled for durations up to the order
%of seconds. Early experiments by Kasevich and Chu using atom
%interferometry of Sodium atoms\cite{Kasevich1992,Kasevich1992a}
%demonstrated the ability to measure gravitational acceleration to a high
%absolute accuracy. Subsequent work also shown the use of atom
%interferometry to measure rotations and reach the long-term stability
%required for inertial navigation~\cite{Durfee2006}. 
%\par\noindent
%When compared to conventional inertial sensors, cold atom systems have
%a much lower measurement bandwidth. This is primarily due to the time taken to prepare a sufficiently large ensemble of atoms.
%During this time, the system is not sensitive to inertial forces.
%Bandwidths of
%\sivalue{60}{\Hz}~\cite{Rakholia2014a}
%and as high as \sivalue{10}{\kHz}~\cite{Biedermann2017} have been
%reported which make use of short loading times. In the former case,
%cold atoms are recaptured after each cycle and in the latter, the
%experiment was performed in a warm vapour without preliminary cooling.
%A high bandwidth necessarily requires a short interferometer pulse
%separation, which results in a reduced acceleration sensitivity.
%For practical applications, the duty cycle must be long enough to ensure that the dynamics of the
%acceleration are not lost during the measurement dead time. 
%\par\noindent
%The concept of hybridising an atom interferometer with a mechanical
%sensor has been discussed in~\cite{Lautier2014}.  
%For instance, there exist commerically available \ac{mems} devices
%which have a noise level of $\sim \sivalue{5}{\mu\g}$ in the
%0-\sivalue{100}{\Hertz} bandwidth. Since the noise between successive
%measurements is uncorrelated, the position error  After many successive measurem
%\section{Application to Inertial Navigation}
%The ability to measure inertial forces, i.e. accelerations and
%rotations, is of
%great practical interest. This is strongly motivated by the need for
%inertial navigation systems. In general, these make use on on-board
%sensors to determine its position using the process of dead reckoning.
%The acceleration and rotation are periodically measured and the
%position of the object relative to its initial position is determined by
%integrating these forces. The accuracy of this method depends upon the intrinsic noise and bias of these devices. These
%lead to a position error which grows in time. For instance, the
%effects of white noise are discussed in~\cite{Wen2016}. In the context
%of this project, the effects of a sensor's stability over time is
%discussed more quantitatively using the Allan variance
%in~\SectionRef{subsec:allan_variance}.
%\par\noindent
%Modern inertial navigation systems reduce the 
%position error through statistical
%techniques such as Kalman filtering~\cite{Faruqi2000}. This produces an unbiased estimate
%of position by taking account of the noise of each sensor. However,
%this is not able to correct for a bias in the sensor. For instance, a constant acceleration bias $\epsilon$ leads to a
%position error that grows quadratically in time. An initial
%calibration of the sensor can remove this bias, but cannot account for
%drift over time. This phenomenon, known as bias instability, is
%the result of effects such as temperature variations and mechanical
%strains. It leads to a characteristic time beyond which the uncertainty is not reduced by averaging successive
%measurements. This
%problem can be alleviated by regularly correcting for the position
%error using an external positioning system, such as \ac{gnss}, and
%re-calibrating the sensors.
\nocite{Dimopoulos2008}
\section{Aims}
This aim of this project was to investigate the use of atom
interferometry in the context of inertial navigation by measuring
horizontal accelerations. Gravitational acceleration induces motion
transverse to the laser wavefront. Together with the requirements of
high acceleration sensitivity and minimal dead-time, this influenced the technical
aspects of this experiment.  
\section{Structure of this Thesis}
This thesis is structured as follows:
\begin{itemize}
  \item \textbf{Chapter 2} presents a theoretical introduction to
    matter-wave interferometry and the relevant atomic structure of
    Rubidium-87.
  \item \textbf{Chapter 3} describes the software used to control the
    experiment and acquire data.
  \item \textbf{Chapter 4} outlines the preliminary cooling and
    trapping of Rubidium-87 in a \ac{mot}.
  \item \textbf{Chapter 5} discusses the techniques used to further
    cool the atoms and prepare an ensemble in the same internal state,
  \item \textbf{Chapter 6} gives a detailed description of the
    in-vacuum optical system for driving Raman transitions.
  \item \textbf{Chapter 7} characterises the interferometer's
    sensitivity to accelerations.
  \item \textbf{Chapter 8} summarises the conclusions of this thesis and outlines
    further work towards improving the experiment
\end{itemize}
%A theoretical introduction to
%matter-wave interferometry and the relevant atomic structure of
%\ac{rb87} is presented in~\ChapterRef{chap:theory}. This is followed
%by a description of the software that was developed to control the experiment and
%acquire data in~\ChapterRef{chap:compinterface}. The subsequent chapters
%focus on various experimental aspects of the project. The preliminary
%cooling and trapping of \ac{rb87} is presented
%in~\ChapterRef{chap:mot}. The techniques used to cool the atoms
%further and prepare an ensemble in the same internal state are
%explained in~\ChapterRef{chap:atom_prep}. This is followed by the
%design and characterisation of an in-vacuum optical system for driving
%Raman transitions in~\ChapterRef{chap:raman_optics}. Finally, a
%characterisation of acceleration-sensitive interference is given
%in~\ChapterRef{chap:atom_int}.



