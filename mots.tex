\chapter{Cooling and Trapping in a MOT}
\section{Chapter Outline}
This chapter presents a description of the components of the experiment which are used to trap and cool atoms in a \ac{mot}. 
\section{The Navigator Vacuum Chamber}\label{sec:vacuum_chamber}
The vacuum chamber, along with the components mounted to it, make up the majority of the hardware used in the preliminary trapping and cooling stages of the experiment. The chamber contains 16 DN40 ConFlat ports arranged on the edges of three octagons, one in each cartesian coordinate plane. Two DN63 ports lie along one axis, which is conventionally taken to be the \(x\) axis. These ports are used to mount the optics for driving Raman transitions and as such, defines the axis along which the atom interferometer is sensitive to accelerations\footnote{For more information about the Raman optical system, refer to \SectionRef{sec:setup_ramanoptics}}. \par\noindent
A diagram of the vacuum chamber and the main \ac{mot} components is shown in \FigureRef{fig:mot_system}. The chamber is pumped down to \ac{uhv} pressures using a NexTorr D100-5 pump. This is a composite system consisting of \ac{neg} and an ion pump. The \ac{neg} is a porous sintered zirconium (St 172) element, which reacts with chemicals such as hydrogen, water, nitrogen, oxygen and hydrocarbons. Most of these were removed during the initial baking and roughing pump stages. Under \ac{uhv} conditions, the largest contributor to the pressure is hydrogen which the \ac{neg} can pump at a speed of \sivalue{100}{\litre\per\second}. Any species that are not absorbed by the \ac{neg}, in particular Rubidium, are pumped by the \sivalue{5}{\litre\per\second} ion pump.
\begin{figure}
    \centering
    \def\svgwidth{1\textwidth}
    \input{Figures/Chapter4/Vacuum.pdf_tex}
    \caption[\ac{mot} system component diagram]{A diagram of the main components on the vacuum chamber used for the \ac{mot} sytems. Rubidium atoms are dispensed and loaded into the 2D \ac{mot} before being pushed into the main chamber and collected in the 3D \ac{mot}. A set of 6 beam collimators provide the light necessary to slow and cool atoms, which are trapped using the spherical quadrupole field generated by the illustrated coils. Not shown are additional bias coils along each \ac{mot} beam axis to null stray fields at the centre of the chamber.}
    \label{fig:mot_system}
\end{figure}   
\subsection{The 2D MOT system}\label{sec:2d_mot}
Initially, atoms were loaded into the 3D \ac{mot} from a background vapour. Whilst this was a relatively simple scheme, a very high partial pressure of Rubidium is required to achieve a fast loading rate (see \SectionRef{subsec:loading_rate}). However, this was undesirable from the point of view of the atom interferometer, as it would result in a poor fringe contrast due to collisions with background atoms and a worse signal to noise ratio, i.e. the ratio of the number of atoms in the interferometer to the total number of detected atoms. Replacing the source of atoms for the 3D \ac{mot} with a side-arm to function as a 2D \ac{mot}~\cite{Dieckmann1998} satisfied the two requirements of a fast loading rate and low base pressure in the main chamber. \par\noindent
A diagram of the setup required to produce a 2D \ac{mot} is presented in~\FigureRef{fig:2d_mot_diagram}. It is similar to the 3D \ac{mot}, with the main exception being that only 4 beams are used to cool the atoms along 2 orthogonal axes. In addition to this, its design is focused towards the production of a large flux of cold atoms which can be subsequently loaded into a 3D \ac{mot}. For instance, the beams used to cool the atoms are collimated to a large waist size and the coils are designed to give a cylindrical quadrupole field with a line of zero magnetic field along the axis of symmetry. Along this axis, the atoms are free to move which results in an atomic beam. To improve the collimation of this atomic beam, a larger radial field gradient than usually used in a 3D \ac{mot} system is used to increase the radial confinement of atoms. In addition, a pinhole is placed at the exit of the cell, so that atoms with a high radial velocity component will miss the aperture. This pinhole also greatly reduces the conductance between the 2D \ac{mot} cell and the main chamber, which means that a comparatively high background pressure (hence, loading rate) can be maintained in the 2D \ac{mot} cell, without greatly increasing the pressure in the main chamber. The pinhole is drilled into a silicon wafer, which is used to partially reflect a beam that propagates along the central axis. This creates an unbalanced molasses that cools atoms with a large axial velocity and due to the differing radiation pressure, pushes atoms out through the pinhole. By slowing a larger proportion of atoms to within the capture velocity of the 3D \ac{mot} and not solely relying on diffusion of the atoms, this configuration, often referred to as a \(2D\+\) \ac{mot}, loads a 3D \ac{mot} faster than the 4-beam counterpart. \par\noindent 
A schematic of the components for the 2D \ac{mot} is presented in~\FigureRef{fig:2d_mot}. The cooling light originates from a single fibre, which is collimated {\huge beam waist size} and linearly polarised before being split into two beams, one for each cooling axis. Each beam passes through a beam-splitter and prism pair, to increase the volume covered by the 2D \ac{mot} beams and is circularly polarised by a pair of \ac{qwp} before entering the \ac{ar} coated glass cell. On the opposing face, a retro-reflecting mirror is used to provide the counter-propagating \ac{mot} beam. This is coated with a layer of quartz to form a \ac{qwp}, so that the reflected beam has the same handedness as the incoming. The cell, manufactured by ColdQuanta, is specifically designed for creating a 2D \ac{mot} and contains two   
\subsection{The 3D MOT system}\label{sec:3d_mot}
\subsection{CCD Imaging}\label{sec:imaging}

\section{Generating MOT light}
\subsection{Muquans Laser Control}
\subsubsection{Frequency Control}
\subsubsection{Real-Time Communication} \label{subsec:muquans_comm}

\section{Controlling the MOTs}
\subsection{Optical Fibre Network}
\subsection{Magnetic Field Control}

\section{Characterising the 3D MOT}
\subsection{3D MOT Loading Rate}\label{subsec:loading_rate}
\subsection{Temperature}

