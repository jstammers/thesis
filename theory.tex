\chapter{Theory}\label{chap:theory}
\section{Chapter Overview}\label{sec:theory_overview}
This chapter presents a theoretical discussion of the interaction
between light and matter that forms the basis of atom interferometry.
It begins with a review of the theory of matter-wave interference and
its application to measuring inertial forces such as accelerations
in~\SectionRef{sec:theory_atomint}. This is followed by a derivation of
the equations of motion for a Raman transition in a simplified
three-level atom
in~\SectionRef{sec:theory_raman}. Finally, this chapter concludes with
a discussion of the atomic structure of \ac{rb87}, focussing on the
aspects relevant to the previously mentioned phenomena.
\par\noindent
This chapter assumes some prior understanding of light-matter
interactions on the part of the reader. A comprehensive description of
the semi-classical approximation to atom-light interactions can be
found in~\cite{Grynberg2011}. Theories of laser cooling and magneto-optical
trapping are of interest for some experimental aspects of these thesis.
Details of these can be found in~\cite{Metcalf2003}. For an exact derivation of
the two-level approximation to the Raman transition, the reader is referred
to~\cite{Wu1996}.  
\section{Light-Pulse Matter-Wave Interference}\label{sec:theory_atomint}
A cornerstone of quantum mechanics is the wave-like evolution of
quantum states as described by the Schr\"odinger equation. As a
consequence of this, it is possible to create superpositions of states
that evolve coherently. Furthermore, these states can interfere with
each other if their wavefunctions overlap. The interference pattern,
which depends on the difference in phase between the two paths.
\par\noindent
In this section, the theory of matter-wave interference using pulses
of laser light is presented. It begins in~\SectionRef{subsec:theory_mz} with a description of the
$\pi/2-\pi-\pi/2$ pulse sequence that forms an analogue of the
Mach-Zehnder interferometer. This is
followed in~\SectionRef{subsec:theory_path} by a derivation of the interferometer phase. In
particular, it is shown that this depends upon the
acceleration of the atom during the interferometer. 

\subsection{The Mach-Zehnder Scheme}\label{subsec:theory_mz}
One method for realising matter-wave interference is the
Ramsey-Bord\'e interferometer~\cite{RBerman1997}. This requires an ensemble
of atoms that can be driven between two states using an interaction
with a laser field. The absorption and stimulated emission of photons
exchanges momentum between the atom and the light such that the state
of the atom is defined as a product of its internal states and
momentum eigenstates, labelled as $\ket{1,\textbf{p}}$ and
$\ket{2,\textbf{p} + \hbar \textbf{k}}$. When compared with
conventional optical interferometry, it is the light which alters the
trajectory of the matter, rather than the other way around. Indeed,
laser pulses with pulse areas of $\pi/2$ and $\pi$ are analogous to
beam-splitters and mirrors in optical systems.  
\par\noindent
A three pulse interferometer is the simplest configuration for which
the two trajectories can overlap after separation. Schematically, this
is shown in~\FigureRef{fig:interferometer}. An atom in
$\ket{1,\textbf{p}}$ is driven into a superposition of two states
using a $\pi/2$ pulse. The wavepacket in $\ket{2,\textbf{p} + \hbar
\textbf{k}}$ has a different momentum, so the two become
spatially separated. After a time $T$, a $\pi$ pulse inverts the state
along each path, so that after another duration of $T$ the two
wavepackets overlap. These are recombined by a final $\pi/2$ pulse.
The interference between the two states is manifested in their
population, which depends on the phase difference between the two
paths. This phase difference $\Phi$ is expressed as follows
\begin{equation}
  \label{eq:phase_difference}
  \Phi = \Phi_\textnormal{prop} +
  \Phi_\textnormal{int}+\Phi_\textnormal{laser}
\end{equation}
where $\Phi_\textnormal{prop}$ is the phase difference due to
propagation along each path, $\Phi_\textnormal{int}$ is the
phase difference due to evolution of the internal state and $
\Phi_\textnormal{laser}$ is the phase difference caused by
interactions with the laser field.   
\begin{figure}[htpb]
  \centering
  \resizebox{0.8\textwidth}{!}{\input{interferometer.pdf_tex}}
  \caption[Mach-Zehnder atom interferometer configuration.]{Mach-Zehnder atom interferometer configuration. A sequence
  of laser pulses are used to drive an atom into a superposition of
two states. When the two paths overlap at the final $\pi/2$ pulse,
they interfere with each other. The final occupation probability is
proportional to the phase difference between the two paths.}
  \label{fig:interferometer}
\end{figure}

\subsection{Phase Shift Contributions}\label{subsec:theory_path}
\subsubsection{Propagation Phase}
The propagation phase can be evaluated using the path integral
approach to quantum mechanics. The trajectory of
a quantum state is obtained by integrating over all possible paths.
The total amplitude for arriving at position $x_b$ at $t_b$, starting
from $x_a$ at $t_a$ is given by the following propagator
\begin{equation}
  K(t_b,x_b,t_a,x_a) = \int_{x_a}^{x_b} e^{i S/\hbar} \mathrm{d}x(t)  
\end{equation}
where $x_b \equiv x(t_b)$ and similarly for $x_a$. The action $S$
along a path is
defined using the Lagrangian of the system
\begin{equation}
  S = \int_{t_a}^{t_b} L(x,\dot{x})\; \mathrm{d}t
  \label{eq:action_def}
\end{equation}
and the classical limit is recovered when the action $S_\textnormal{cl}
\gg
\hbar$. In this case, most paths destructively interfere as their
phases oscillate rapidly. Along the classical path, the action is
minimised. Paths close to this lead to constructive interference, so
the trajectory of the system is well-described using the Euler-Lagrange equations. Under this
condition, it can be shown~\cite{Storey1994} that for a plane wave,
the propagation phase is proportional to the action along the
classical path
\begin{equation}
  \label{eq:prop_phase}
  \Phi_\textnormal{prop} = \frac{1}{\hbar} S_\textnormal{cl}(t_b,x_b,t_a,x_a)
\end{equation}
\par\noindent
This is indeed the case for an atom interferometer. When considering a
single atom under a constant
acceleration, the Lagrangian is given by
\begin{equation}
  \label{eq:lagrange_acc}
  L = \frac{1}{2}m\dot{x}^2 + m a x
\end{equation}
so that it's position and velocity are given by
\begin{subequations}
\begin{align}
  x(t) &= x_a + v_a (t-t_a) + \frac{1}{2} a (t-t_a)^2 \\
  v(t) &= v_a + a(t-t_a)
\end{align}
\end{subequations}
for initial values of position ($x_a$), velocity ($v_a$) and time
($t_a$).     
Using~\EquationRef{eq:action_def}, the action is given by 
\begin{equation}
  \label{eq:action_classical}
  S(t_b,x_b,t_a,x_a) = \frac{m(x_b-x_a)^2}{2(t_b-t_a)} + \frac{m a
  (x_b+x_a)(t_b-t_a)}{2} -\frac{m a^2 (t_b-t_a)^3}{24} 
\end{equation}
The trajectories along each interferometer path under acceleration are
shown in~\FigureRef{fig:interferometer_accel}. Denoting
$S_\textnormal{ac} \equiv S(T,x_c,0,x_a)$ and likewise for the other
co-ordinates, the difference in the action along the upper and lower paths is 
\begin{align}
  S_\textnormal{ac} + S_\textnormal{cd} -
\left(S_\textnormal{ab}+S_\textnormal{bd}\right) &= -\frac{\text{m}
(x_b-x_c) \left(a T^2- x_a+x_b+x_c-x_d\right)}{T}\nonumber \\
&= -\frac{\text{m}
(x'_b-x'_c) \left((x'_c-x'_a) - (x'_d-x'_b)\right)}{T}
\label{eq:prop_action}
\end{align}
The second expression is obtained using $x_b - x_b' = x_c - x_c'
= \frac{1}{2}a T^2$ and $x_d-x_d' = 2 a T^2$, where the primes
indicate positions in the absence of acceleration. The last factor on
the numerator of~\EquationRef{eq:prop_action} is zero because the
paths enclose a parallelogram for which $x_c' - x_a' = x_d'-x_b'$. It follows
from~\EquationRef{eq:prop_phase} that there is no difference in the
propagation phase along the two paths $acd \;
(\phi_\textnormal{prop}^{u})$ and $abd \; (\phi_\textnormal{prop}^{l})$. In other words,
$\Phi_\textnormal{prop} \equiv \phi_\textnormal{prop}^{u}-
\phi_\textnormal{prop}^{l}=
0$ when there is a constant acceleration of the atom.
\begin{figure}[htpb]
  \centering
  \resizebox{0.8\textwidth}{!}{\input{interferometer_accel.pdf_tex}}
  \caption[Interferemeter paths under constant
  acceleration]{Interferometer paths under a constant acceleration.
  The dashed lines indicate the trajectories under zero acceleration.}
\label{fig:interferometer_accel}
\end{figure}
\subsubsection{Internal State Evolution}
Between each laser pulse, the internal state of the atom evolves freely. Therefore, the
phase due to this evolution is given by
\begin{equation}
  \label{eq:internal_evolution}
  \phi^{(j)}(t_a,t_b) = \int_{t_a}^{t_b} \omega_j t\;\mathrm{d} t
\end{equation}
where $\hbar \omega_j$ is the internal energy of the atom along that
trajectory. The phase difference due to the internal state evolution
is 
\begin{equation}
  \label{eq:phase_internal}
  \Phi_\textnormal{int} = \phi^{(2)}(t_1,t_2) +
  \phi^{(1)}(t_2,t_3) - \left(\phi^{(1)}(t_1,t_2) + \phi^{(2)}(t_2,t_3)\right)
\end{equation}
which is zero if the time between successive pulses is the same and
the energy of each state does not vary. The possibility of energy
variation is important
when considering the effects of the ac Stark shift and is addressed
subsequently in~\SectionRef{subsec:light_shift}.

\subsubsection{Laser Phase}\label{subsec:laser_phase}
Finally, there is the contribution from the laser phase. During each
transition, the phase of the laser modifies the state of the atom. The
propagator that describes this transition for arbitrary Rabi
frequencies and detuning is derived below,
in~\SectionRef{sec:theory_raman}. For now, it is sufficient to focus
on the ideal case of perfect pulse areas and zero detuning. The first
and third pulses have pulse areas of $\pi/2$, which modifies the two
states as follows
\begin{subequations}
  \begin{align}
    U_{\pi/2}(\phi) \ket{1} &= \frac{1}{\sqrt{2}}\ket{1}+ 
    \frac{e^{-i \phi}}{\sqrt{2}} \ket{2} \\
    U_{\pi/2}(\phi) \ket{2} &=  \frac{e^{i \phi}}{\sqrt{2}} \ket{1}
    + \frac{1}{\sqrt{2}} \ket{2}
  \end{align}
\end{subequations}
where $\phi = \textbf{k.x} - \omega_\text{L}t + \phi_0$ is the phase
of the light driving the transition in the atom at position
$\textbf{x}$. The middle pulse has a $\pi$ pulse area whose interaction is
described by
\begin{subequations}
  \begin{align}
    U_\pi(\phi) \ket{1} &= e^{-i \phi} \ket{2} \\
    U_\pi(\phi) \ket{2} &= e^{i \phi} \ket{1} 
  \end{align}
\end{subequations}
At the output corresponding to the state $\ket{1}$, the phases along the upper and lower path are then 
\begin{subequations}
  \begin{align}
    \phi_l &= \textbf{k}.(\textbf{x}_d-\textbf{x}_b) - \omega_\text{L}
    T + (\phi_{0})_{3} -(\phi_{0})_{2}  \\
    \phi_u &= \textbf{k}.(\textbf{x}_c-\textbf{x}_a) - \omega_\text{L}
    T + (\phi_{0})_{2} -\left(\phi_{0}\right)_{1}  \\
  \end{align}
\end{subequations}
where $(\phi_{0})_{j}$ refers to the phase of the laser at the $j-$th
light pulse. Following the same argument used in~\EquationRef{eq:prop_action}, this
can be simplified so that the phase difference due to the laser interaction ($\phi_u -
\phi_l$), and hence the total phase difference, is
\begin{equation}
  \label{eq:phase_interferometer}
  \Phi_\textnormal{laser} = \textbf{k.a} T^2 - (\phi_0)_{1} + 2(\phi_0)_{2}- (\phi_{0})_{3}
\end{equation}
In summary, the phase difference along the two paths of the
interferometer comes not from the atomic motion along the path, nor
from the phase evolution of the internal states, but rather the
position-dependent phase of the light field, which is impressed on the
atom each time a transition is driven. In other words, $\Phi =
\Phi_\textnormal{laser}$. This makes a phase difference
of $\textbf{k.a} T^2$. In addition, if the phase of the light is
adjusted between pulses, there is a further interferometer phase of
$- (\phi_0)_{1} + 2(\phi_0)_{2}- (\phi_{0})_{3}$. This second part
vanishes if the phase of the light is undisturbed between pulses.
\section{Raman Transitions in Rubidium-87}\label{sec:theory_raman}
In this section, an analysis of the dynamics of the stimulated Raman transition is
presented. This is modeled as an effective transition between two
ground states $\ket{1}$ and $\ket{2}$. These two states are coupled
via an intermediate state $\ket{i}$ which can be adiabatically
eliminated when the two light fields are sufficiently detuned from
resonance. An explicit derivation of this process can be found
elsewhere
\cite{Weiss1994, Han2012}. The key results, which aid the discussion
of the performance of the interferometer in subsequent chapters, are
summarised below. Following this, a propagator is derived which
describes the evolution of the atomic state during a Raman transition.
This is used to derive the final state after the interferometer pulse
sequence. Modeling the atomic system in this way makes it possible to consider
the effects of finite pulse duration, detuning and intensity
on the final state. 
\par\noindent
The energy level scheme for a Raman transition is shown
in~\FigureRef{fig:raman_model}. Each ground state is coupled to the
intermediate state $\ket{i}$ by an interaction with an electric field  
$ \textbf{E}_j=
\textbf{E}_{0,j} \cos(\omega_j^l t - \textbf{k}_j.\textbf{x}+
\phi_j)$. The strength of this interaction is defined by the Rabi
frequency\begin{equation}
  \label{eq:rabi_freq}
  \Omega_j = \frac{1}{\hbar}\bra{j} -\textbf{d}.\textbf{E}_{0,j}
  \ket{i} e^{i \phi_j}
\end{equation}
where $\textbf{d}$ is the dipole operator. Referring
to~\FigureRef{fig:raman_model}, it is assumed that $\omega_\text{hfs} \gg \Delta \gg \delta$, so
that $\textbf{E}_1$ couples only to state $\ket{1}$ and $\textbf{E}_2$
couples only to state $\ket{2}$. 
%The total electric field is described by the superposition $\textbf{E}_1 + \textbf{E}_2$. 
When $\Delta$ is sufficiently large, the frequency of the two fields
are far off-resonant from their corresponding transition frequencies
that couple \trans{j}{i}. In this case, the population of the
intermediate state remains small. This state can be adiabatically
eliminated to result in an effective two-level system. The
interference of the two fields means that the transition can be
represented by an interaction with an effective field with the
following parameters
\begin{subequations}
  \begin{align}
    \omega_\textnormal{eff} &= \omega^l_1 - \omega^l_2 \\
    \phi_\textnormal{eff} &= \phi_1 -\phi_2 \\
    \keff &= \textbf{k}_1 - \textbf{k}_2 \\
          &\approx 2\textbf{k}_1\;
    \text{(because the beams will be counter-propagating)} \nonumber 
  \end{align}
\end{subequations}
The resonance condition for the Raman transition is given by 
\begin{equation}
  \delta = \omega_\textnormal{eff} - \left[\omega_\textnormal{hfs} + \keff .
\textbf{v} + \frac{\hbar |\keff|^2}{2m}\right]
  \label{eq:raman_detuning}
\end{equation}
Here, the second term in the square brackets is the Doppler shift and
the third is the recoil shift.
\begin{figure}[htpb]
  \centering
  \fontsize{16pt}{16pt}
  \resizebox{0.4\textwidth}{!}{\input{raman_model.pdf_tex}}
  \caption[Schematic diagram of a raman transition.]{Schematic diagram of a raman transition. Two states,
  $\ket{1}$ and $\ket{2}$, are coupled to each other via
  an
  intermediate state $\ket{i}$. When $\Delta$ is large enough (see
  text) $\ket{i}$ can be adiabatically eliminated, resulting in an
effective two-level system.}
  \label{fig:raman_model}
\end{figure}
\subsection{State Propagation}
To describe the dynamics of the Raman transition, it is convenient to
express the system in the interaction picture. This allows the
Hamiltonian to be represented in a time-independent form. An analytic
solution to the equations of motion can then be obtained, which
describes the evolution of the atomic state during a Raman transition. After adiabatically
eliminating the intermediate state~\cite{Weiss1994}, the state of the
atom in a rotating frame is given by
\begin{equation}
  \label{eq:state_raman}
  \ket{\psi} = c_1 e^{-i \omega_1 t}\ket{1} + c_2 e^{-i (\omega_2 +
  \delta) t} \ket{2}
\end{equation}
The Hamiltonian can be simplified by making the rotating wave
approximation, which neglects the rapidly oscillating terms. After
this, the Schr\"odinger equation becomes
\begin{equation}
  \label{eq:raman_eom}
  i \begin{pmatrix} \dot{c}_1 \\ \dot{c}_2 \end{pmatrix} =
  \frac{1}{2}\begin{pmatrix}
    -\Omega_1^\textnormal{ac} & \Omega_\textnormal{eff} \\
    \Omega_\textnormal{eff}^*&  -2\delta - \Omega_2^\textnormal{ac}  \\
    \end{pmatrix} \begin{pmatrix} c_1 \\ c_2\end{pmatrix}
\end{equation}
Where in terms of the single-photon Rabi frequencies $\Omega_1$ and
$\Omega_2$, the ac Stark shifts $\Omega_j^\textnormal{ac}$ and
effective Rabi frequency $\Omega_\textnormal{eff}$ are given by 
\begin{subequations}
\begin{align}
  \Omega_j^\textnormal{ac} &= -\frac{|\Omega_j|^2}{2 \Delta
  }\label{eq:raman_ac_stark}\\
  \Omega_\textnormal{eff}&=\frac{e^{i\phi_\textnormal{eff}} |\Omega_1|
    |\Omega_2|}{2 \Delta }\label{eq:raman_rabi} 
\end{align}
\end{subequations}
where the contribution of $\delta$ to the one-photon detuning of the
\trans{2}{i} transition has been neglected compared with $\Delta$.
Under an appropriate choice of intensities and two-photon detuning, it
is possible to cancel the differential ac Stark shift
$\Omega_1^\textnormal{ac} -\Omega_2^\textnormal{ac}$ (see
\SectionRef{subsec:light_shift} for further details on this).
Therefore, they can be omitted from~\EquationRef{eq:raman_eom} so that
after solving the Schr\"odinger equation, the coefficients $c_1,c_2$
evolve according to 
\begin{equation}
  \label{eq:schrodinger_prop}
  \begin{pmatrix} c_1(t) \\ c_2(t)\end{pmatrix} = U \begin{pmatrix}
c_1(0) \\ c_2(0)\end{pmatrix}
\end{equation}
where the propagator $U$ is
\begin{equation}
  U = e^{\frac{1}{2}i\delta t } \begin{pmatrix} 
  \cos \left(\frac{ \Omega' t}{2}\right)-i\frac{\delta}{\Omega'}\sin
  \left(\frac{\Omega' t}{2}\right) & 
 -i e^{i \phi_\textnormal{eff}} \frac{
 \Omega_\textnormal{eff}}{\Omega'} \sin \left(\frac{\Omega'
 t}{2}\right) \\ 
  -i e^{-i \phi_\textnormal{eff}} \frac{
  \Omega_\textnormal{eff}}{\Omega'} \sin \left(\frac{\Omega'
t}{2}\right)& \cos \left(\frac{ \Omega' t}{2}\right)+i\frac{
\delta  }{\Omega'}\sin \left(\frac{ \Omega' t}{2}\right) \\
\end{pmatrix}
\label{eq:propagator_first}
\end{equation}
and $\Omega' = \sqrt{\delta^2 + \Omega_\textnormal{eff}^2}$ is the
generalised Rabi frequency~\cite{Ramsey1950}. After transforming back into the
stationary frame, \EquationRef{eq:propagator_first} becomes
\begin{equation}
U =  \begin{pmatrix} 
  e^{\frac{i(\omega_\textnormal{eff}-\delta_R)t}{2}  }\left(\cos \left(\frac{ \Omega' t}{2}\right)-i\frac{\delta}{\Omega'}\sin
  \left(\frac{\Omega' t}{2}\right)\right) & 
 -i e^{\frac{i(\omega_\textnormal{eff}-\delta_R)t}{2}  }e^{i \phi_\textnormal{eff}} \frac{
 \Omega_\textnormal{eff}}{\Omega'} \sin \left(\frac{\Omega'
 t}{2}\right) \\ 
  -i e^{\frac{-i(\omega_\textnormal{eff}-\delta_R)t}{2}  }e^{-i \phi_\textnormal{eff}} \frac{
  \Omega_\textnormal{eff}}{\Omega'} \sin \left(\frac{\Omega'
  t}{2}\right)& e^{-\frac{i(\omega_\textnormal{eff}-\delta_R)t}{2}  }\left(\cos \left(\frac{ \Omega' t}{2}\right)+i\frac{
\delta  }{\Omega'}\sin \left(\frac{ \Omega' t}{2}\right)\right) \\
\end{pmatrix}
\label{eq:propagator_stationary}
\end{equation}  
where $\delta_R = \keff.\textbf{v} + \hbar |\keff|^2/2m$ is the Raman
detuning from
\EquationRef{eq:raman_detuning}.
\subsection{Application to a Sequence of Raman Pulses}
Now, this propagator can be used to determine the state of the atom
after a sequence of Raman pulses. The propagator was derived using an
initial state at $t=0$. This can be generalised to an arbitrary time
by replacing $t\rightarrow t-t_0$ and $\phi_\textnormal{eff} \rightarrow \phi_\textnormal{eff} +
\omega_\textnormal{eff} t_0$. Between successive Raman pulses, there is a period of
free evolution for a time $T$. During this, the state evolves
according to
\begin{equation}
  \label{eq:free_evolution}
  F(T) = \begin{pmatrix} e^{-i \frac{\omega_\textnormal{hfs} T}{2}} &0 \\
  0 & e^{i \frac{\omega_\textnormal{hfs} T}{2}}\end{pmatrix}
\end{equation}
Treating the Rabi frequency, phase, detuning and pulse duration as free
parameters denoted by an index $j=1,2,3$, the state after three arbitrary pulses separated in time
is given by
\begin{equation}
  \label{eq:general_state}
  \ket{\psi} = U(\Omega_3,\phi_3,\delta_3,\tau_3) F(T_2)
  U(\Omega_2,\phi_2,\delta_2,\tau_2) F(T_1)
  U(\Omega_1,\phi_1,\delta_1,\tau_1)
  \ket{\psi_0}
\end{equation}
for an initial state $\ket{\psi_0}$. This expression is valid for
arbitrary pulses and is useful when considering a
distribution of intensities and Doppler detunings across an atomic
ensemble. For instance, the occupation probability for $\ket{2}$ is
given by 
\begin{equation}
  \label{eq:occ_prob}
  P_{\ket{2}} = |\braket{2|\psi}|^2
\end{equation}
It is also useful to define the fringe contrast -- the difference
between the maximum and minimum of $P_{\ket{2}}$
\begin{equation}
  \label{eq:fringe_contrast}
  c = \max_{\Phi} P_{\ket{2}} -  \min_{\Phi} P_{\ket{2}} 
\end{equation}
which ranges between 0 and 1. This fringe contrast implicitly depends
on the pulse area and detuning of each pulse. In the case of perfect pulse areas and zero
detuning with the initial state $\ket{\psi_0} =
\ket{1}$, \EquationRef{eq:occ_prob} yields the following
\begin{align}
  \label{eq:state_simp}
  P'_{\ket{2}} &= \sin \left(\frac{1}{2} (\phi_1-2
  \phi_2+\phi_3)\right)^2 \nonumber \\
  &= \sin\left(\frac{\Phi}{2}\right)^2,
\end{align}
where $\Phi$ is related to the acceleration of the atom
in~\EquationRef{eq:phase_interferometer}.
%In order to observe interference between two atomic states, it is
%essential that they remain coherent over the required timescale. The
%most suitable choice of states for this are the $\ket{F=1,m_F=0}$ and
%$\ket{F=2,m_F=0}$ hyperfine ground states. Neither of these states
%will spontaneously decay and their energy has only a second-order
%dependence on magnetic fields. However, the transition frequency
%$\omega_\text{hfs} \approx 2\pi \times$ \sivalue{6.835}{\GHz} is
%stimulated using microwave radiation, so each photon imparts little
%momentum to the atom. Since the sensitivity of an atom interferometer to
%inertial forces is proportional to the momentum transferred, this can
%be increased by stimulating Raman transitions using light at optical
%frequencies. In this case, an atom will absorb a photon from one light
%field and emit into the other. If the two fields are
%counter-propagating, this has the effect 
\subsection{Extending to a Real Atomic System}

In a real atomic system, there are many intermediate states which
couple to the two ground states. This theory can be extended to
consider the additional intermediate states using their individual
Rabi frequencies. In this case, the effective Rabi frequency is given
by the following sum
\begin{equation}
  \label{eq:rabi_raman_sum}
  \Omega_\textnormal{eff} = e^{i \phi_\textnormal{eff}}  \sum_k
\frac{|\Omega_{1k}| |\Omega_{2k}|}{2\Delta_k}
\end{equation}
where the index $k$ labels the intermediate states which couple to
both $\ket{1}$ and $\ket{2}$. A similar expression can be obtained for
the ac Stark shift terms. Note that the previous assumption that
$\textbf{E}_1$ only couples to the state $\ket{1}$, and similarly for
$\textbf{E}_2$, is no longer valid. A full treatment includes the
coupling of each field to each state. The effective Rabi frequency in \EquationRef{eq:rabi_raman_sum} implicitly
depends on the polarisation of the light. The individual Rabi
frequencies $\Omega_{jk}$ are calculated using the Clebsch-Gordan
coefficients that describe the coupling of angular momentum states.
Further detail on calculating transition matrix elements of tensor
operators can be found in~\cite{Brink1968}.  
%\par\noindent
%A Raman transition is a two-photon process whereby the state of an
%atom evolves by interacting with two laser fields. This can occur in
%an atomic system with two states $\ket{1}$ and $\ket{2}$ that are both coupled to an
%intermediate state $\ket{i}$ via electric dipole transitions. The eigenstates
%$\ket{j,\textbf{p}_j}$ are products of the internal state $\ket{j}$ and a momentum
%eigenstate $\ket{\textbf{p}_j} = e^{i
%\frac{\textbf{p}_j.\textbf{x}}{\hbar}}$. In general,
%the state of the atom is a superposition of these eigenstates
%\begin{equation}
%  \ket{\psi} = c_1\ket{1,\textbf{p}_1} + c_2
%  \ket{2,\textbf{p}_2} + c_i \ket{i,\textbf{p}_i}
%  \label{eq:raman_state}
%\end{equation}
%and the Hamiltonian that describes the evolution of this system is
%\begin{equation}
%  H = H_0 + H_I
%  \label{eq:raman_hamiltonian}
%\end{equation}
%where $H_0$ is the free Hamiltonian
%\begin{equation}
%  H_0 = \hbar \begin{pmatrix}
%    \omega_1 + \frac{p_1^2}{2m\hbar}& 0 & 0 \\
%    0 & \omega_2+ \frac{p_2^2}{2m\hbar} & 0 \\
%    0 & 0 & \omega_i+ \frac{p_i^2}{2m\hbar}
%  \end{pmatrix}
%  \label{eq:free_hamiltonian}
%\end{equation}
%and $H_I$ is the interaction Hamiltonian which describes the
%interaction between the atom and the light. Writing the
%electric fields of the respective lasers as $ \textbf{E}_j=
%\textbf{E}_{0,j} \cos(\omega_j^l t - \textbf{k}_j.\textbf{x}+
%\phi_j)$ for $j=1,2$, the interaction Hamiltonian depends upon
%their superposition $\textbf{E}_1 + \textbf{E}_2$. Assuming that the
%energy difference of $\ket{1}$ and $\ket{2}$ is much greater than the
%transition linewidth, each field stimulates transitions from just one of the lower states. In this case, $H_I$ is  
%\begin{equation}
%  \label{eq:interaction_hamiltonian}
%  H_I = \frac{\hbar}{2} \begin{pmatrix}
%    0 & 0 & \Omega_1e^{-i (\omega_1^l t- \textbf{k}_1.\textbf{x}+\phi_1)} \\
%    0 & 0 &\Omega_2 e^{-i (\omega_2^l t- \textbf{k}_2.\textbf{x}+\phi_2)} \\
%    \Omega^*_1 e^{i (\omega_1^l t - \textbf{k}_1.\textbf{x}+ \phi_1)} & \Omega_2^* e^{i
%    (\omega_2^l t - \textbf{k}_2.\textbf{x}+ \phi_2)}& 0
%  \end{pmatrix} + c.c. 
%\end{equation}
%where $\Omega_j$ is the Rabi frequency
%\begin{equation}
%  \label{eq:rabi_freq}
%  \Omega_j = \bra{j} -\textbf{d}.\textbf{E}_{0,j} \ket{i}
%\end{equation}
%and $\textbf{d}$ is the electric dipole operator. 
%In the interaction picture, the state of the atom is given by
%\begin{equation}
%  \ket{\psi(t)} = c_1 e^{-\gamma_1 t} \ket{1, \textbf{p}_1} +c_2 e^{-i
%  \gamma_2 t} \ket{2, \textbf{p}_2} +c_i e^{-i \gamma_i t} \ket{i, \textbf{p}_i} 
%  \label{eq:raman_state_time}
%\end{equation}
%where $\gamma_i \equiv \omega_i + p_i^2/2m\hbar$. The interaction
%Hamiltonian becomes
%\begin{align}
%  \label{eq:hamiltonian_interaction}
%  H_{I,I} &= e^{i H_0 t/\hbar} H_I e^{-i H_0 t/\hbar} \nonumber \\
% &=  \frac{\hbar}{2} \begin{pmatrix}
%  0 & 0 &e^{i \gamma_{1i} t} \Omega_1e^{-i (\omega_1^l t- \textbf{k}_1.\textbf{x}+\phi_1)} \\
%    0 & 0 &e^{i \gamma_{2i} t}\Omega_2 e^{-i (\omega_2^l t- \textbf{k}_2.\textbf{x}+\phi_2)} \\
%    e^{-i \gamma_{1i} t}\Omega^*_1 e^{i (\omega_1^l t -
%      \textbf{k}_1.\textbf{x}+ \phi_1)} & e^{-i \gamma_{2i} t}\Omega_2^* e^{i
%    (\omega_2^l t - \textbf{k}_2.\textbf{x}+ \phi_2)}& 0
%  \end{pmatrix} + c.c.
%\end{align}
%where $\gamma_{ji} \equiv \gamma_j - \gamma_i$. The terms oscillating
%at a fast frequency (implicitly defined as $c.c.$) can now be
%neglected using the rotating wave approximation as over the timescale
%of the interaction, their contribution will average to zero. 
%\begin{equation}
%  \ket{\psi(t)} = c_1(0) e^{-i \omega_1 t} \ket{1} +c_2(0) e^{-i \omega_2 t} \ket{2} +c_i(0) e^{-i \omega_i t} \ket{i} 
%  \label{eq:raman_state_time}
%\end{equation}this system is described by the following
%hamiltonian
\
\section{Applications to Rubidium-87}
This final section places what has been discussed thus far into the
context of Rubidium-87, the atomic species used in this experiment. It
begins in~\SectionRef{subsec:theory_atomic} with a presentation of the hyperfine structure in the
$5S_{1/2}$ and $5P_{3/2}$ levels, used throughout this experiment for
cooling and trapping, as well as interferometry. This is followed by an overview of the
techniques used to drive counter-propagating Raman transitions
in~\SectionRef{subsec:theory_rb87_raman}. In
particular, this motivates the choice of polarisation used for the two
Raman beams. Finally, this section concludes in~\SectionRef{sec:theory_double_int} with a discussion of the two interferometer paths. 
\subsection{Atomic Structure}\label{subsec:theory_atomic}
The $5S_{1/2}$ electronic ground state of Rubidium-87 has two
hyperfine levels $F=1$ and $F=2$, split by \sivalue{6.8}{\GHz}. 
It can be efficiently laser
cooled using one laser (the cooler) to scatter light from atoms in
$F=2$, and another (the repumper) to pump atoms out of $F=1$ and into
$F=2$.
\par\noindent
In addition to these, two hyperfine
levels in its ground state provide suitable choices of states for
interferometry. Neither decays 
spontaneously, which helps to preserve their coherence during
interferometry. Particular pairs of Zeeman sub-levels can be
identified which are coupled via a Raman transition. When the two
laser fields are counter-propagating, the photon recoil imparts $10^5$
times more momentum
than the one-photon \sivalue{6.8}{\GHz} microwave transition. This is desirable for
acceleration sensing, since the sensitivity is directly proportional to
the recoil momentum. 
\par\noindent
The D2 transition between the $5S_{1/2}$ ground state and $5P_{3/2}$
excited state is
shown in~\FigureRef{fig:rb87_hyperfine}. The key transitions used in
this experiment are also indicated. The Zeeman sub-levels (with
magnetic quantum numbers $m_F$ ranging from $-F$ to $F$) are not
shown explicitly. Their energies are shifted in a magnetic field
$B$ by an amount $g_F m_F B$, where $g_F$ is the Land\'e $g$-factor.
The values listed are taken
from~\cite{Steck2001}. Further quantities, such as relative transition
strengths and physical properties of Rubidium-87 are contained
therein. 
\begin{figure}[htpb]
  \centering
  \resizebox{0.8\textwidth}{!}{\input{Figures/Chapter2/rb87_hyperfine.pdf_tex}}
  \caption[Rubidium-87 D2 transition hyperfine structure.]{\ac{rb87} D2 transition hyperfine structure. The absolute
    energy difference of the $\ket{F=2}$ and $\ket{F'=3}$ levels is shown
    as an equivalent frequency. The energy of the other levels are
    shown relative to this. The
  approximate $g$-factors for each hyperfine level and Zeeman
shifts are also indicated. The main transitions used in this
experiment are indicated by red and blue arrows. The values are taken
from \cite{Steck2001}.} 
  \label{fig:rb87_hyperfine}
\end{figure}
\subsection{Driving Raman Transitions}\label{subsec:theory_rb87_raman}
Within the Rubidium-87 ground state, there are multiple pairs of
states that can be coherently coupled using a Raman transition.
A natural choice is for interferometry is to use the two $m_F = 0$ sub-levels\footnote{In what follows,
  the states $\ket{1}$ and $\ket{2}$ will be used as alternative
  notation for $\ket{F=1,m_F=0}$ and $\ket{F=2,m_F=0}$ in instances
  where the
explicit quantum numbers are not required}. These
are magnetically-insensitive; their transition frequency has at most a
second-order Zeeman shift. Consequently, this transition is least
affected by magnetic field variations from field gradients or
environment noise. 
\par\noindent
The polarisation of the two light fields is
crucial to ensuring that the $\Delta m_F = 0$ transition is driven.
The $\ket{1,0} \rightarrow \ket{2,0}$ transition does not change the
magnetic quantum number, i.e. the projection of angular momentum
along the given quantisation axis. However, the total angular momentum quantum number
increases by 1, which means that the Raman transition can be driven
when the two electric
fields are orthogonally linearly polarised (lin $\perp$ lin) to each
other and to the quantisation axis. \par\noindent
For circular polarisation, it can be shown that the two fields must
have the same handedness. The electric field of a right-handed
circularly polarised beam is $\textbf{E} = \textbf{E}_0\left(\mathcal{E}_- e^{-i \omega t} + \mathcal{E}_+
e^{i \omega t}\right)$, where $\mathcal{E}_-$ is the spherical tensor
component that lowers the angular momentum of the light and
$\mathcal{E}_+$ raises it. The product $r \equiv \textbf{d.E}$
can be written as
\begin{equation}
  \label{eq:dipole_expansion}
  r_j =  
E_0(d^{(j)}_+ \mathcal{E}^{(j)}_- e^{-i \omega_j t} + d^{(j)}_- \mathcal{E}^{(j)}_+ e^{i
\omega_j t})
\end{equation}
where $d^{(j)}_+$ and $d^{(j)}_-$ are the components of the dipole operator which
increase and decrease $m_F$, respectively. The index $j$ labels
the $\ket{j,0}$ state. The product $r_1 r_2$ is expanded in the
rotating wave approximation to
give
\begin{equation}
  \label{eq:dipole_exp}
  r_1 r_2= E^{(1)}_0E^{(2)}_0(d^{(1)}_+ \mathcal{E}^{(1)}_- d^{(2)}_-
  \mathcal{E}^{(2)}_+ e^{-i (\omega_1 - \omega_2)
    t} + d^{(1)}_- \mathcal{E}^{(1)}_+ d^{(2)}_+ \mathcal{E}^{(2)}_-  e^{i(
\omega_1 -\omega_2)t})
\end{equation}
where the first term corresponds to the component which drives
$\ket{1,0} \rightarrow \ket{2,0}$ and the second to $\ket{2,0}
\rightarrow \ket{1,0}$.
The terms proportional
to $(\omega_1 + \omega_2)$ have been neglected as they are
off-resonant. A similar argument follows in the case of two left-handed
circular polarised fields $\textbf{E} =
\textbf{E}_0\left(\mathcal{E}_+ e^{-i \omega t} - \mathcal{E}_-
e^{i \omega t}\right)$. Conversely, the transition is not driven if one beam
is left-handed and the other is right-handed, since the near-resonant product of dipole operators
results in components of the form $d^{(1)}_+ d^{(2)}_+$, which
change the magnetic quantum number by $\pm2$.
\subsection{The Double Interferometer}\label{sec:theory_double_int}
The two frequencies for driving the Raman transition are sent into the
experiment along the fast and slow axis of a \ac{pm} fibre. 
To produce the necessary counter-propagating beams, a \ac{qwp} and
mirror retro-reflect and invert the polarisation of each beam. This
results in the beam configuration shown in~\FigureRef{fig:double_int}.
With a magnetic field defining a $z-$axis parallel to the light's
wavevector, the fields are all driving $\sigma^\pm$ transitions. The
co-propagating $\sigma^+ - \sigma^-$ combinations cannot drive the
$\Delta m_F = 0$ \trans{1,0}{2,0} transition, leaving the two
counter-propagating $\sigma^{+} - \sigma^{+}$ and $\sigma^{-} -
\sigma^{-}$ combinations. These have effective wavevectors in opposite
directions, which means that an atom resonant with both pairs is
driven into a superposition of $\ket{1,\textbf{p}},
  \ket{2,\textbf{p}+\hbar \textbf{k}}$ and  $\ket{2,\textbf{p}-\hbar
  \textbf{k}}$. The $\pi/2-\pi-\pi/2$ pulse sequence results in a trajectory shown schematically
  in~\FigureRef{fig:double_int_path}. Some of the paths close to
  form two interferometers with oppositely signed phase shifts, making
  the interferometer insensitive to acceleration.
\begin{figure}[htpb]
  \centering
  \resizebox{0.8\textwidth}{!}{\input{double_int.pdf_tex}}
  \caption[Raman beam configuration.]{Raman beam configuration. The two incoming beams are
  orthogonally circularly polarised. The reflected beams pass through
a \ac{qwp} to invert their handedness. The selection rules of the
Raman transition result in two pairs of beams that drive transitions
with oppositely directed effective wavevectors.}
  \label{fig:double_int}
\end{figure}
Fortunately, these Raman transitions have Doppler shifts of
opposite sign. Therefore, by appropriately detuning the light from
resonance, an atom with a non-zero velocity will be shifted closer to
resonance with one pair and further out from the other. Indeed, if
the difference between their resonances $2 |\keff.\textbf{v}|$ is
larger than the transition linewidth, only one pair of beams drives
the transition. The moving molasses method used to launch the atoms
and lift this degeneracy is
discussed later, in~\SectionRef{subsec:moving_molasses}; a characterisation of the Raman
transition spectrum in the experiment is presented
in~\SectionRef{subsec:raman_spec} 
\begin{figure}[htpb]
  \centering
  \fontsize{18pt}{18pt}
  \resizebox{0.4\textwidth}{!}{\input{double_int_path.pdf_tex}}
  \caption[Interferometer paths with two oppositely directed Raman
  transitions.]{Interferometer paths with two oppositely directed Raman
  transitions. If the atom is resonant with both transitions, its
momentum changes by $\pm \hbar \keff$. This results in two
interferometers with oppositely signed phase differences. The dashed
directed lines indicate trajectories which will not interfere, leading
to a loss of coherence.}
\label{fig:double_int_path}
\end{figure}


\section{Conclusion}
This chapter has presented a theoretical overview of matter-wave
interferometry and motivated the use of a $\pi/2-\pi-\pi/2$ sequence
of laser pulses to measure the acceleration of an atom. Furthermore,
it was shown how this can be implemented using a Raman transition
between two ground states in
\ac{rb87}. This induces a larger momentum recoil when compared to a
microwave transition and thus, a greater sensitivity to accelerations.
The effect of rotation on the interferometer phase has not been
discussed. For further details on an atom interferometer gyroscope,
the reader is referred to~\cite{Gauguet2009}.
\nocite{Dubetsky2006}
\nocite{Han2012}
